\documentclass[a4paper]{article}

% Table of contents depth (currently unused)
\setcounter{tocdepth}{3}

% Section numbering depth (zero for no numbering)
\setcounter{secnumdepth}{0}

% latex package inclusions here
%\usepackage{fullpage}
\usepackage{hyperref}
\usepackage{tabulary}
%\usepackage{amsthm}

% set up BNF generator
%\usepackage{syntax}
%\setlength{\grammarparsep}{10pt plus 1pt minus 1pt}
%\setlength{\grammarindent}{10em} 

% set up source code inclusion
\usepackage{listings}
\lstset{
  tabsize=2,
  basicstyle = \ttfamily\small,
  columns=fullflexible
}
% Usage for the above like so:
% \begin{lstlisting}
%   CODE CODE CODE
% \end{lstlisting}

% in-line code styling (same style as listing)
\newcommand{\shell}[1]{\lstinline{#1}}

% Line and paragraph spacing
\newenvironment{linewise}
  {\parindent=0pt
   \obeyspaces\obeylines
   \begingroup\lccode`~=`\^^M
   \lowercase{\endgroup\def~}{\par\leavevmode}}
  {\ignorespacesafterend}

%%%%%%%%%%%%%%%%%%%%%%%%%%%%%%%%%%%%%%%%%%%%%%%%%%%%%%%%%%%%%%%%%%%%%%%%%%%%%%%

\begin{document}
\title{Beowulf Notes}
\date{2016}
\author{
Daniel Clay \\ 
}
\maketitle

%%%%%%%%%%%%%%%%%%%%%%%%%%%%%%%%%%%%%%%%%%%%%%%%%%%%%%%%%%%%%%%%%%%%%%%%%%%%%%%
\section{Introduction}
%%%%%%%%%%%%%%%%%%%%%%%%%%%%%%%%%%%%%%%%%%%%%%%%%%%%%%%%%%%%%%%%%%%%%%%%%%%%%%%

\subsection{Introduction \& Omissions}%%%%%%%%%%%%%%%%%%%%%%%%%%%%%%%%%%%%%%%%%

\textit{Beowulf} is an Anglo-Saxon epic poem, probably written at some point 
between the 6th and 8th centuries and set in 6th century Denmark. The original
is some 3000 lines long, and is written in a primarily West Saxon (IE from Wessex)
dialect of Old English.

To adapt such a poem for the stage requires a good deal of changes to be made
to the script, changes which are detailed in the accompanying document
'Beowulf - Additions and Omissions'. This document aims to provide a set of
'director's notes' to accompany the script, linking the actions in the script
with the passages describing them and providing hints or explanations of details.

\subsection{Notation}%%%%%%%%%%%%%%%%%%%%%%%%%%%%%%%%%%%%%%%%%%%%%%%%%%%%%%%%%%

Parts of the play are referenced by their scene, while lines from the poem are
numbered from the Michael Alexander translation rather than the original.

%%%%%%%%%%%%%%%%%%%%%%%%%%%%%%%%%%%%%%%%%%%%%%%%%%%%%%%%%%%%%%%%%%%%%%%%%%%%%%%
\section{Notes accompanying the script}
%%%%%%%%%%%%%%%%%%%%%%%%%%%%%%%%%%%%%%%%%%%%%%%%%%%%%%%%%%%%%%%%%%%%%%%%%%%%%%%

\linewise{

\subsection{Act 1}%%%%%%%%%%%%%%%%%%%%%%%%%%%%%%%%%%%%%%%%%%%%%%%%%%%%%%%%%%%%%

\centerline{\textbf{Scene 1 - Introduction}}

The poem opens with a story about Scyld Shefing, founder of the royal house
of the Spear-Danes, of which \textit{Hrothgar} is a member. \textit{The Bard} 
goes through much of the back story of the Syclding dynasty, the building of Heorot
and introduces Grendel and Beowulf before the play proper begins.

The introduction is taken almost verbatim from the poem, merely missing out sections
for brevity; such as the passage describing Scyld's burial.

While most of the contemporary audience would have been familiar with the story
described (hence the Bard's use of the phrase 'you have heard'), it serves as
a useful introduction to the culture for a modern audience. 

The long monologue from the Bard also mirrors the way the poem would have been
performed.

\centerline{\textbf{Scene 2 - Beowulf's Arrival at Heorot}}

The play itself opens with \textit{Beowulf's} arrival at Heorot (his departure
from Geatland and subsequent arrival in Denmark having been narrated, while his
conversation with the lookout has been omitted). 

The transition between narration and performance occurs at the moment Beowulf
and his companions enter: suddenly the story is made flesh. 

The next few passages are almost entirely dialogue and cover the ritual 
greeting between host and guest; king and thegn. 

Much of what has been cut describes small details - how Wulfgar is considered wise,
and the quality of Beowulf's mail shirt. 

After the greetings and Beowulf's request to be allowed to fight Grendel, a bench
is cleared for the visitors and the feasting commences. 

This is described in the poem as follows:

\textit{A bench was then cleared for the company of Geats
there in the beer-hall, for the whole band together.
The stout-hearted warriors went to their places,
bore their strength proudly. Prompt in his office,
the man who held the horn of bright mead
poured out its sweetness. The song of the poet
again rang in Heorot. The heroes laughed loud
in the great gathering of the Geats and the Danes.}

\centerline{\textbf{Scene 3 - Feasting and Bragging}}

\centerline{\textbf{Scene 4 - Grendel}}

The fight with Grendel covers the entirety of lines 720-824 and goes into considerable detail.

My performance directions have been kept brief; the actors should study the following passage
to work out exactly how they should perform the fight:

\textit{ Walking to the hall came this warlike creature
condemned to agony. The door gave way,
toughened with iron, at the touch of those hands.
Rage-inflamed, wreckage-bent, he ripped open
the jaws of the hall. Hastening on,
the foe then stepped onto the unstained floor,
angrily advanced: out of his eyes stood
an unlovely light like that of fire.
He saw then in the hall a host of young soldiers,
a company of kinsmen caught away in sleep,
a whole warrior-band. In his heart he laughed then,
horrible monster, his hopes swelling
to a gluttonous meal. He meant to wrench
the life from each body that lay in the place
before night was done. It was not to be;
he was no longer to feast on the flesh of mankind
after that night.

Narrowly the powerful
kinsman of Hygelac kept watch how the ravager
set to work with his sudden catches;
nor did the monster mean to hang back.
As a first step he set his hands on
a sleeping soldier, savagely tore at him,
gnashed at his bone-joints, bolted huge gobbets,
sucked at his veins, and had soon eaten
all of the dead man, even down to his
hands and feet.

Forward he stepped,
stretched out his hands to seize the warrior
calmly at rest there, reached out for him with his
unfriendly fingers: but the faster man
forestalling, sat up, sent back his arm.

The upholder of evils at once knew
he had not met, on middle earth’s
extremest acres, with any man
of harder hand-grip: his heart panicked.
He was quit of the place no more quickly for that.
Eager to be away, he ailed for his darkness
and the company of devils; the dealings he had there
were like nothing he had come across in his lifetime.

Then Hygelac’s brave kinsman called to mind
that evening’s utterance, upright he stood,
fastened his hold till fingers were bursting.
The monster strained away: the man stepped closer.
The monster’s desire was for darkness between them,
direction regardless, to get out and run
for his fen-bordered lair; he felt his grip’s strength
crushed by his enemy. It was an ill journey
the rough marauder had made to Heorot.

The crash in the banqueting-hall came to the Danes,
the men of the guard that remained in the buildings,
with the taste of death. The deepening rage
of the claimants to Heorot caused it to resound.
It was indeed wonderful that the wine-supper-hall
withstood the wrestling pair, that the world’s palace
fell not to the ground. But it was girt firmly,
both inside and out, by iron braces
of skilled manufacture. Many a figured
gold-worked wine-bench, as we heard it,
started from the floor at the struggles of that pair.
The men of the Danes had not imagined that
any of mankind by what method soever
might undo that intricate, antlered hall,
sunder it by strength – unless it were swallowed up in
the embraces of fire.

Fear entered into
the listening North Danes, as that noise rose up again
strange and strident. It shrilled terror
to the ears that heard it through the hall’s side-wall,
the grisly plaint of God’s enemy,
his song of ill-success, the sobs of the damned one
bewailing his pain. He was pinioned there
by the man of all mankind living
in this world’s estate the strongest of his hands.

Not for anything would the earls’ guardian
let his deadly guest go living:
he did not count his continued existence
of the least use to anyone. The earls ran
to defend the person of their famous prince;
they drew their ancestral swords to bring
what aid they could to their captain, Beowulf.

They were ignorant of this, when they entered the fight,
boldly-intentioned battle-friends,
to hew at Grendel, hunt his life
on every side – that no sword on earth,
not the truest steel, could touch their assailant;
for by a spell he had dispossessed all
blades of their bite on him.

A bitter parting
from life was that day destined for him;
the eldritch spirit was sent off on his
far faring into the fiends’ domain.
It was then that this monster, who, moved by spite
against human kind, had caused so much harm
– so feuding with God – found at last
that flesh and bone were to fail him in the end;
for Hygelac’s great-hearted kinsman
had him by the hand; and hateful to each
was the breath of the other.

A breach in the giant
flesh-frame showed then, shoulder-muscles
sprang apart, there was a snapping of tendons,
bone-locks burst. To Beowulf the glory
of this fight was granted; Grendel’s lot
to flee the slopes fen-ward with flagging heart,
to a den where he knew there could be no relief,
no refuge for a life at its very last stage,
whose surrender-day had dawned. The Danish hopes
in this fatal fight had found their answer.}

It should be noted that despite the lack of actual dialogue,
the scene should still be noisy - in particular Grendel's sobs 
and shrill cries. 

If desired, Beowulf can ad lib some lines or use the lines below when challenging
Grendel before they begin wrestling:

\textbf{Beowulf} Grendel! Here stands undaunted
 a warrior who will defend Hrothgar's land!

(Alternatively, in the Old English)

\textit{ Grendel! her stynt unforcuð
þe wile gealgean Hrothgar eard!}

\centerline{\textbf{Scene 5 - Grendel's Death is Celebrated}}

With regard to the gift giving, the exact gifts given are not so important.
What is important is the idea that Hrothgar's generosity is enormous, and the gifts
given extremely fine.

\subsection{Act 2}%%%%%%%%%%%%%%%%%%%%%%%%%%%%%%%%%%%%%%%%%%%%%%%%%%%%%%%%%%%%%

\centerline{\textbf{Scene 1 - Grendel's Mother Attacks Heorot}}

\centerline{\textbf{Scene 2 - The Warriors Take Stock}}

\centerline{\textbf{Scene 3 - Journey to the Mere}}

\centerline{\textbf{Scene 4 - Grendel's Mother}}

\centerline{\textbf{Scene 5 - The Hero Returns}}

\centerline{\textbf{Scene 6 - More Celebrations}}

\centerline{\textbf{Scene 7 - The Geats Take Their Leave}}

\centerline{\textbf{Scene 8 - Return to Hygelac's Court}}

\subsection{Interval}%%%%%%%%%%%%%%%%%%%%%%%%%%%%%%%%%%%%%%%%%%%%%%%%%%%%%%%%%%

\subsection{Act 3}%%%%%%%%%%%%%%%%%%%%%%%%%%%%%%%%%%%%%%%%%%%%%%%%%%%%%%%%%%%%%

\centerline{\textbf{Scene 1 - The Dragon Awakens}}

\centerline{\textbf{Scene 2 - The Dragon}}

\centerline{\textbf{Scene 3 - Beowulf Dies}}

\centerline{\textbf{Scene 4 - Grief}}

\centerline{\textbf{Scene 5 - The Dirge}}

Unlike the rest of the text; this scene is perfomed in the old English, it is sung 
rather than spoken, and it is taken from a different work.

The song which closes the performance is a fragment of the Old English poem \textit{The Wanderer}.

}

%%%%%%%%%%%%%%%%%%%%%%%%%%%%%%%%%%%%%%%%%%%%%%%%%%%%%%%%%%%%%%%%%%%%%%%%%%%%%%%
\section{Characters}
%%%%%%%%%%%%%%%%%%%%%%%%%%%%%%%%%%%%%%%%%%%%%%%%%%%%%%%%%%%%%%%%%%%%%%%%%%%%%%%

\end{document}
