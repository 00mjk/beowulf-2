\documentclass[a4paper]{article}

% Table of contents depth (currently unused)
\setcounter{tocdepth}{3}

% Section numbering depth (zero for no numbering)
\setcounter{secnumdepth}{0}

% latex package inclusions here
%\usepackage{fullpage}
\usepackage{hyperref}
\usepackage{tabulary}
%\usepackage{amsthm}

% set up BNF generator
%\usepackage{syntax}
%\setlength{\grammarparsep}{10pt plus 1pt minus 1pt}
%\setlength{\grammarindent}{10em} 

% set up source code inclusion
\usepackage{listings}
\lstset{
  tabsize=2,
  basicstyle = \ttfamily\small,
  columns=fullflexible
}
% Usage for the above like so:
% \begin{lstlisting}
%   CODE CODE CODE
% \end{lstlisting}

% in-line code styling (same style as listing)
\newcommand{\shell}[1]{\lstinline{#1}}

% Line and paragraph spacing
\newenvironment{linewise}
  {\parindent=0pt
   \obeyspaces\obeylines
   \begingroup\lccode`~=`\^^M
   \lowercase{\endgroup\def~}{\par\leavevmode}}
  {\ignorespacesafterend}

%%%%%%%%%%%%%%%%%%%%%%%%%%%%%%%%%%%%%%%%%%%%%%%%%%%%%%%%%%%%%%%%%%%%%%%%%%%%%%%

\begin{document}
\title{Beowulf - Additions \& Omissions}
\date{2016}
\author{
Daniel Clay \\ 
}
\maketitle

%%%%%%%%%%%%%%%%%%%%%%%%%%%%%%%%%%%%%%%%%%%%%%%%%%%%%%%%%%%%%%%%%%%%%%%%%%%%%%%
\section{Introduction}
%%%%%%%%%%%%%%%%%%%%%%%%%%%%%%%%%%%%%%%%%%%%%%%%%%%%%%%%%%%%%%%%%%%%%%%%%%%%%%%

\subsection{Overview}%%%%%%%%%%%%%%%%%%%%%%%%%%%%%%%%%%%%%%%%%%%%%%%%%%%%%%%%%%

While adapting Michael Alexander's verse translation into a play; I realised that it would 
be necessary to cut or change a substantial amount of the play. This document
aims to explain what was changed, and why. 

A poem, by it's very nature, has fewer limitations than a stage play. One of these
is the ability to gloss over actions or dialogue that in a play must be performed
in their entirety. For all of these instances, I have been forced to write dialogue
to fill the gaps suggested by the poet. I have tried my best to match the style of 
the original, and have noted in this document every insertion.

Likewise, poems can contain elements that would be difficult to stage, or severely
disrupt the flow of the plot. For the former I have employed the device of a narrator,
to describe those events that cannot be staged - for example the journey from Heorot
to the Mere. For the latter, I have cut them as much as possible leaving only what
is directly relevant to the plot. The account of Scyld Shefing's ship burial has been 
entirely cut, and the following genealogy reduced to it's bare essentials.

The narrator has also been employed to introduce and end scenes, keeping the
story moving. I hope that the transition between that which is performed, and
that which is narrated will be seamless.

\subsection{Notation}%%%%%%%%%%%%%%%%%%%%%%%%%%%%%%%%%%%%%%%%%%%%%%%%%%%%%%%%%%

Each scene in the play has been given it's own section, and within each scene I 
have listed all the additions and omissions which are clearly labelled. Line
numbers relate to the line numbers in the Michael Alexander translation, not
the original.

\newpage

%%%%%%%%%%%%%%%%%%%%%%%%%%%%%%%%%%%%%%%%%%%%%%%%%%%%%%%%%%%%%%%%%%%%%%%%%%%%%%%
\section{Script - Additions and Omissions}
%%%%%%%%%%%%%%%%%%%%%%%%%%%%%%%%%%%%%%%%%%%%%%%%%%%%%%%%%%%%%%%%%%%%%%%%%%%%%%%

\subsection{Act 1}%%%%%%%%%%%%%%%%%%%%%%%%%%%%%%%%%%%%%%%%%%%%%%%%%%%%%%%%%%%%%

\linewise{

\centerline{\textbf{Scene 1}}

\centerline{\textbf{Scene 2}}

\centerline{\textbf{Scene 3}}

\centerline{\textbf{Scene 4}}

\centerline{\textbf{Scene 5}}

\subsection{Act 2}%%%%%%%%%%%%%%%%%%%%%%%%%%%%%%%%%%%%%%%%%%%%%%%%%%%%%%%%%%%%%

\centerline{\textbf{Scene 1}}

\centerline{\textbf{Scene 2}}

\centerline{\textbf{Scene 3}}

\centerline{\textbf{Scene 4}}

\centerline{\textbf{Scene 5}}

\centerline{\textbf{Scene 6}}

\centerline{\textbf{Scene 7}}

\subsection{Interval}%%%%%%%%%%%%%%%%%%%%%%%%%%%%%%%%%%%%%%%%%%%%%%%%%%%%%%%%%%

\subsection{Act 3}%%%%%%%%%%%%%%%%%%%%%%%%%%%%%%%%%%%%%%%%%%%%%%%%%%%%%%%%%%%%%

\centerline{\textbf{Scene 1}}

\centerline{\textbf{Scene 2}}

\centerline{\textbf{Scene 3}}

\centerline{\textbf{Scene 4}}

\centerline{\textbf{Scene 5}}

}

\end{document}
