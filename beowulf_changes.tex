\documentclass[a4paper]{article}

% Table of contents depth (currently unused)
\setcounter{tocdepth}{3}

% Section numbering depth (zero for no numbering)
\setcounter{secnumdepth}{0}

% latex package inclusions here
%\usepackage{fullpage}
\usepackage{hyperref}
\usepackage{tabulary}
%\usepackage{amsthm}

% set up BNF generator
%\usepackage{syntax}
%\setlength{\grammarparsep}{10pt plus 1pt minus 1pt}
%\setlength{\grammarindent}{10em} 

% set up source code inclusion
\usepackage{listings}
\lstset{
  tabsize=2,
  basicstyle = \ttfamily\small,
  columns=fullflexible
}
% Usage for the above like so:
% \begin{lstlisting}
%   CODE CODE CODE
% \end{lstlisting}

% in-line code styling (same style as listing)
\newcommand{\shell}[1]{\lstinline{#1}}

% Line and paragraph spacing
\newenvironment{linewise}
  {\parindent=0pt
   \obeyspaces\obeylines
   \begingroup\lccode`~=`\^^M
   \lowercase{\endgroup\def~}{\par\leavevmode}}
  {\ignorespacesafterend}

%%%%%%%%%%%%%%%%%%%%%%%%%%%%%%%%%%%%%%%%%%%%%%%%%%%%%%%%%%%%%%%%%%%%%%%%%%%%%%%

\begin{document}
\title{Beowulf - Additions \& Omissions}
\date{2016}
\author{
Daniel Clay \\ 
}
\maketitle

%%%%%%%%%%%%%%%%%%%%%%%%%%%%%%%%%%%%%%%%%%%%%%%%%%%%%%%%%%%%%%%%%%%%%%%%%%%%%%%
\section{Introduction}
%%%%%%%%%%%%%%%%%%%%%%%%%%%%%%%%%%%%%%%%%%%%%%%%%%%%%%%%%%%%%%%%%%%%%%%%%%%%%%%

\subsection{Overview}%%%%%%%%%%%%%%%%%%%%%%%%%%%%%%%%%%%%%%%%%%%%%%%%%%%%%%%%%%

While adapting Michael Alexander's verse translation into a play; I realised that it would 
be necessary to cut or change a substantial amount of the play. This document
aims to explain what was changed, and why. 

A poem, by it's very nature, has fewer limitations than a stage play. One of these
is the ability to gloss over actions or dialogue that in a play must be performed
in their entirety. For all of these instances, I have been forced to write dialogue
to fill the gaps suggested by the poet. I have tried my best to match the style of 
the original, and have noted in this document every insertion.

Likewise, poems can contain elements that would be difficult to stage, or severely
disrupt the flow of the plot. For the former I have employed the device of a narrator,
to describe those events that cannot be staged - for example the journey from Heorot
to the Mere. For the latter, I have cut them as much as possible leaving only what
is directly relevant to the plot. The account of Scyld Shefing's ship burial has been 
entirely cut, and the following genealogy reduced to its' bare essentials.

The poem also contains numerous references to other stories. Several of these
are told by bards within Heorot, and so if desired could be sung as part of the 
background noise during the appropriate scenes.

Lastly, in a few places I have cut dialogue for brevity. This has been done only
in a few scenes, where the character has veered off on a tangent and all dialogue
is otherwise faithful to the translation.

The narrator has also been employed to introduce and end scenes, keeping the
story moving. I hope that the transition between that which is performed, and
that which is narrated will be seamless.

\subsection{Notation}%%%%%%%%%%%%%%%%%%%%%%%%%%%%%%%%%%%%%%%%%%%%%%%%%%%%%%%%%%

Each scene in the play has been given it's own section, and within each scene I 
have listed all the changes which should be easily understandable. Line
numbers relate to the line numbers in the \textit{Michael Alexander} translation, not
the original.

\newpage

%%%%%%%%%%%%%%%%%%%%%%%%%%%%%%%%%%%%%%%%%%%%%%%%%%%%%%%%%%%%%%%%%%%%%%%%%%%%%%%
\section{Script - Additions and Omissions}
%%%%%%%%%%%%%%%%%%%%%%%%%%%%%%%%%%%%%%%%%%%%%%%%%%%%%%%%%%%%%%%%%%%%%%%%%%%%%%%

\subsection{Act 1}%%%%%%%%%%%%%%%%%%%%%%%%%%%%%%%%%%%%%%%%%%%%%%%%%%%%%%%%%%%%%

\linewise{

\centerline{\textbf{Scene 1 - Introduction}}

The play opens with a condensed version of the first 324 lines of the poem. 
The genealogy of Hrothgar is cut down to the barest minimum, and the passage
concerning the burial of Scyld Shefing is almost completely removed.

Also removed is the Geats' arrival in Denmark and their conversation with the
coastguard. I felt that this passage didn't add anything to the story, and
since virtually everything in it is repeated when they arrive at Heorot I 
decided to omit it.

\centerline{\textbf{Scene 2 - Beowulf's Arrival at Heorot}}

There is a bit of a continuity error here - at line 328 the Geats put down their
spears and shields. At line 387 Wulfgar tells them to leave their weapons before
going to the king. It is unclear whether Wulfgar is telling them to down their weapons,
or telling them not to take them up again; since I have omitted the passage which
describes them placing them down, I have decided to have the Geats leave their weapons
after Wulfgar's instruction.

\centerline{\textbf{Scene 3 - Feasting and Bragging}}

Beowulf and Unferth's conversation proceeds as in the poem, before Wealhtheow arrives.
Her arrival is described in the poem (lines 611-630), but her lines are paraphrased.
All of her dialogue for this fragment was written by me, to fit the following lines:

\textit{There was laughter of heroes, harp-music ran,
words were warm-hearted. Wealhtheow moved,
mindful of courtesies, the queen of Hrothgar,
glittering to greet the Geats in the hall,
peerless lady; but to the land’s guardian
she offered first the flowing cup,
bade him be blithe at the beer-drinking,
gracious to his people; gladly the conqueror
partook of the banquet, tasted the hall-cup.

The Helming princess then passed about among
the old and the young men in each part of the hall,
bringing the treasure-cup, until the time came
when the flashing-armed queen, complete in all virtues,
carried out to Beowulf the brimming vessel;
she greeted the Geat, and gave thanks to the Lord
in words wisely chosen, her wish being granted
to meet with a man who might be counted on
for aid against these troubles. He took then the cup,
a man violent in war, at Wealhtheow’s hand,
and framed his utterance, eager for the conflict.}

\centerline{\textbf{Scene 4 - Grendel}}

As most of this scene as written is dialogue free, I have paraphrased the prose into
stage directions, and taken the liberty of writing some lines for Beowulf which are
loosely based on Earldorman Byrhtnoth's challenge in \textit{The Battle of Maldon}.

\centerline{\textbf{Scene 5 - Grendel's Death is Celebrated}}

After the fight, lines 836- 873 describe how Grendel is tracked back to the Mere, while
lines 874-919 tell of Sigemund. Both have been omitted.

In addition to the banner, mail shirt, helmet and sword given to Beowulf,
Hrothgar also gives eight horses and saddles. These I have omitted as being
hugely impractical to stage.

While giving the gifts, Hrothgar is described as telling Beowulf to 'use them well'.
I have written dialogue to this effect.

After the gift giving is another passage of exposition, telling of Finn's sons.
Like all the unrelated stories in the poem, I have omitted it.

Hrethic and Hrothmund, the sons of Hrothgar have no lines and only feature in this scene.
For that reason I have not included them in the cast, although if desired they 
could be added in with little change to the play. If included, Beowulf should
be seated between them.

Wealtheow then gives Beowulf arm rings, robes, rings and a 'rich collar'. I
have abbreviated this list to just the arm rings and necklance for practicality. 
While doing this, she speaks some lines which are only hinted at; I have written some.

After this, the poet is briefly sidetracked talking about a collar stolen by Hama;
this has been omitted.

The passage introducing Grendel's Mother, from 1250-1277, has been cut down somewhat.

\subsection{Act 2}%%%%%%%%%%%%%%%%%%%%%%%%%%%%%%%%%%%%%%%%%%%%%%%%%%%%%%%%%%%%%

\centerline{\textbf{Scene 1 - Grendel's Mother Attacks Heorot}}

No changes made.

\centerline{\textbf{Scene 2 - The Warriors Take Stock}}

Beowulf's address to Hrothgar upon being summoned to the hall
is paraphrased in the poem; I have written a couple of lines.

Likewise, I have written a couple of lines for Hrothgar at the
end of the scene.

\centerline{\textbf{Scene 3 - Journey to the Mere}}

On arrival at the Mere, Beowulf kills a sea snake with his bow. 
This I omitted as being difficult to stage and irrelevant to the plot.

Beowulf and Unferth then have a conversation during which Unferth lends
Beowulf his sword. Because this is paraphrased in the poem, I have written
dialogue for them. 

\centerline{\textbf{Scene 4 - Grendel's Mother}}

No changes made.

\centerline{\textbf{Scene 5 - The Hero Returns}}

I have cut lines 1605-1622 because they describe what happens when
Beowulf cuts of Grendel's head - this I have included in the previous scene 
as part of his fight with Grendel's Mother.

\centerline{\textbf{Scene 6 - More Celebrations}}

No changes made.

\centerline{\textbf{Scene 7 - The Geats Take Their Leave}}

As before, the coastguard's part has been cut.

Lines 1929 - 1960 tell of Modthryth and Offa, and
being irrelevant to the plot, have been cut. 

\centerline{\textbf{Scene 8 - Return to Hygelac's Court}}

I have cut lines 2019-2068 - in them the poet goes off on 
a tangent talking about Freawaru and Ingeld.

Line 2079 refers to Handscio being eaten whole. This would be
impractical to stage so has been removed.

Lines 2162-2165 refer to the horses given by Hrothgar. Since 
I cut these (Act 1 Scene 5), they have been removed.

I have changed the ordering of the end of the scene somewhat;
Hygelac's presentation of gifts has been moved forward slightly
so that the Bard's text happens all together at the end of the act.

As Hrothgar's words are paraphrased in the poem, I have written some dialogue.
The original passage is as follows:

\textit{ Then the king bold in war, keeper of the warriors,
required them to bring in the bequest of Hrethel,
elaborate in gold; the Geats at that day
had no more royal treasure of the rank of sword.
This he then laid in the lap of Beowulf
and bestowed on him an estate of seven thousand hides,
a chief’s stool and a hall.}

\subsection{Interval}%%%%%%%%%%%%%%%%%%%%%%%%%%%%%%%%%%%%%%%%%%%%%%%%%%%%%%%%%%

\subsection{Act 3}%%%%%%%%%%%%%%%%%%%%%%%%%%%%%%%%%%%%%%%%%%%%%%%%%%%%%%%%%%%%%

\centerline{\textbf{Scene 1 - The Dragon Awakens}}

\centerline{\textbf{Scene 2 - The Dragon}}

\centerline{\textbf{Scene 3 - Beowulf Dies}}

\centerline{\textbf{Scene 4 - Grief}}

The play proper closes with most of lines 2842-3179 of the poem.
Lines 2842-2860 are performed, rather than spoken.  
Wiglaf's speech to the cowards (lines 2861-2888) is performed in it's entirety, 
after which the narrator takes over and recites lines 2889-2906, before skipping 
the poet's predictions of doom for the Geats, which takes up lines 2907- 3003.

The narrator resumes at line 3004, and recites the lines attributed to the 
messenger up to line 3034. The narration then skips lines 3035-3073 and Wiglaf takes up 
the story once more.

His speech, covering lines 3074-3106 is included in full, with a brief interjection
from the narrator before continuing from lines 3111-3116.

The rest of the play proper, covering lines 3117-3179 of the poem, is entirely 
narrated without change from the poem.


\end{document}
