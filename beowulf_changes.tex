\documentclass[a4paper]{article}

% Table of contents depth (currently unused)
\setcounter{tocdepth}{3}

% Section numbering depth (zero for no numbering)
\setcounter{secnumdepth}{0}

% latex package inclusions here
%\usepackage{fullpage}
\usepackage{hyperref}
\usepackage{tabulary}
%\usepackage{amsthm}

% set up BNF generator
%\usepackage{syntax}
%\setlength{\grammarparsep}{10pt plus 1pt minus 1pt}
%\setlength{\grammarindent}{10em} 

% set up source code inclusion
\usepackage{listings}
\lstset{
  tabsize=2,
  basicstyle = \ttfamily\small,
  columns=fullflexible
}
% Usage for the above like so:
% \begin{lstlisting}
%   CODE CODE CODE
% \end{lstlisting}

% in-line code styling (same style as listing)
\newcommand{\shell}[1]{\lstinline{#1}}

% Line and paragraph spacing
\newenvironment{linewise}
  {\parindent=0pt
   \obeyspaces\obeylines
   \begingroup\lccode`~=`\^^M
   \lowercase{\endgroup\def~}{\par\leavevmode}}
  {\ignorespacesafterend}

%%%%%%%%%%%%%%%%%%%%%%%%%%%%%%%%%%%%%%%%%%%%%%%%%%%%%%%%%%%%%%%%%%%%%%%%%%%%%%%

\begin{document}
\title{Beowulf - Additions \& Omissions}
\date{2016}
\author{
Daniel Clay \\ 
}
\maketitle

%%%%%%%%%%%%%%%%%%%%%%%%%%%%%%%%%%%%%%%%%%%%%%%%%%%%%%%%%%%%%%%%%%%%%%%%%%%%%%%
\section{Introduction}
%%%%%%%%%%%%%%%%%%%%%%%%%%%%%%%%%%%%%%%%%%%%%%%%%%%%%%%%%%%%%%%%%%%%%%%%%%%%%%%

\subsection{Overview}%%%%%%%%%%%%%%%%%%%%%%%%%%%%%%%%%%%%%%%%%%%%%%%%%%%%%%%%%%

While adapting Michael Alexander's verse translation into a play; I realised that it would 
be necessary to cut or change a substantial amount of the play. This document
aims to explain what was changed, and why. 

A poem, by it's very nature, has fewer limitations than a stage play. One of these
is the ability to gloss over actions or dialogue that in a play must be performed
in their entirety. For all of these instances, I have been forced to write dialogue
to fill the gaps suggested by the poet. I have tried my best to match the style of 
the original, and have noted in this document every insertion.

Likewise, poems can contain elements that would be difficult to stage, or severely
disrupt the flow of the plot. For the former I have employed the device of a narrator,
to describe those events that cannot be staged - for example the journey from Heorot
to the Mere. For the latter, I have cut them as much as possible leaving only what
is directly relevant to the plot. The account of Scyld Shefing's ship burial has been 
entirely cut, and the following genealogy reduced to it's bare essentials.

Lastly, in a few places I have cut dialogue for brevity. This has been done only
in a few scenes, where the character has veered off on a tangent and all dialogue
is otherwise faithful to the translation.

The narrator has also been employed to introduce and end scenes, keeping the
story moving. I hope that the transition between that which is performed, and
that which is narrated will be seamless.

\subsection{Notation}%%%%%%%%%%%%%%%%%%%%%%%%%%%%%%%%%%%%%%%%%%%%%%%%%%%%%%%%%%

Each scene in the play has been given it's own section, and within each scene I 
have listed all the changes which should be easily understandable. Line
numbers relate to the line numbers in the \textit{Michael Alexander} translation, not
the original.

\newpage

%%%%%%%%%%%%%%%%%%%%%%%%%%%%%%%%%%%%%%%%%%%%%%%%%%%%%%%%%%%%%%%%%%%%%%%%%%%%%%%
\section{Script - Additions and Omissions}
%%%%%%%%%%%%%%%%%%%%%%%%%%%%%%%%%%%%%%%%%%%%%%%%%%%%%%%%%%%%%%%%%%%%%%%%%%%%%%%

\subsection{Act 1}%%%%%%%%%%%%%%%%%%%%%%%%%%%%%%%%%%%%%%%%%%%%%%%%%%%%%%%%%%%%%

\linewise{

\centerline{\textbf{Scene 1 - Introduction}}

The play opens with a condensed version of the first 324 lines of the poem.

\centerline{\textbf{Scene 2 - Beowulf's Arrival at Heorot}}

There is a bit of a continuity error here - at line 328 the Geats put down their
spears and shields. At line 387 Wulfgar tells them to leave their weapons before
going to the king. It is unclear whether Wulfgar is telling them to down their weapons,
or telling them not to take them up again; since I have omitted the passage which
describes them placing them down, I have decided to have the Geats leave their weapons
after Wulfgar's instruction.

\centerline{\textbf{Scene 3 - Feasting and Bragging}}

\centerline{\textbf{Scene 4 - Grendel}}

\centerline{\textbf{Scene 5 - Grendel's Death is Celebrated}}

\subsection{Act 2}%%%%%%%%%%%%%%%%%%%%%%%%%%%%%%%%%%%%%%%%%%%%%%%%%%%%%%%%%%%%%

\centerline{\textbf{Scene 1 - Grendel's Mother Attacks Heorot}}

\centerline{\textbf{Scene 2 - The Warriors Take Stock}}

\centerline{\textbf{Scene 3 - Journey to the Mere}}

\centerline{\textbf{Scene 4 - Grendel's Mother}}

\centerline{\textbf{Scene 5 - The Hero Returns}}

\centerline{\textbf{Scene 6 - More Celebrations}}

\centerline{\textbf{Scene 7 - The Geats Take Their Leave}}

\centerline{\textbf{Scene 8 - Return to Hygelac's Court}}

\subsection{Interval}%%%%%%%%%%%%%%%%%%%%%%%%%%%%%%%%%%%%%%%%%%%%%%%%%%%%%%%%%%

\subsection{Act 3}%%%%%%%%%%%%%%%%%%%%%%%%%%%%%%%%%%%%%%%%%%%%%%%%%%%%%%%%%%%%%

\centerline{\textbf{Scene 1 - The Dragon Awakens}}

\centerline{\textbf{Scene 2 - The Dragon}}

\centerline{\textbf{Scene 3 - Beowulf Dies}}

\centerline{\textbf{Scene 4 - Grief}}

The play proper closes with most of lines 2842-3179 of the poem.
Lines 2842-2860 are performed, rather than spoken.  
Wiglaf's speech to the cowards (lines 2861-2888) is performed in it's entirety, 
after which the narrator takes over and recites lines 2889-2906, before skipping 
the poet's predictions of doom for the Geats, which takes up lines 2907- 3003.

The narrator resumes at line 3004, and recites the lines attributed to the 
messenger up to line 3034. The narration then skips lines 3035-3073 and Wiglaf takes up 
the story once more.

His speech, covering lines 3074-3106 is included in full, with a brief interjection
from the narrator before continuing from lines 3111-3116.

The rest of the play proper, covering lines 3117-3179 of the poem, is entirely 
narrated without change from the poem.

\centerline{\textbf{Scene 5 - The Dirge}}

Unlike the rest of the text; this scene is perfomed in the old English, it is sung 
rather than spoken, and it is taken from a different work.

The song which closes the performance is a fragment of the Old English poem \textit{The Wanderer}.

The original is 119 lines long, and the fragment included covers lines 92-109 inclusive.

}

\end{document}
