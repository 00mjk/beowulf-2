\documentclass[a4paper]{article}

% Table of contents depth (currently unused)
\setcounter{tocdepth}{3}

% Section numbering depth (zero for no numbering)
\setcounter{secnumdepth}{0}

% latex package inclusions here
%\usepackage{fullpage}
\usepackage{hyperref}
\usepackage{tabulary}
%\usepackage{amsthm}

% set up BNF generator
%\usepackage{syntax}
%\setlength{\grammarparsep}{10pt plus 1pt minus 1pt}
%\setlength{\grammarindent}{10em} 

% set up source code inclusion
\usepackage{listings}
\lstset{
  tabsize=2,
  basicstyle = \ttfamily\small,
  columns=fullflexible
}
% Usage for the above like so:
% \begin{lstlisting}
%   CODE CODE CODE
% \end{lstlisting}

% in-line code styling (same style as listing)
\newcommand{\shell}[1]{\lstinline{#1}}

% Line and paragraph spacing
\newenvironment{linewise}
  {\parindent=0pt
   \obeyspaces\obeylines
   \begingroup\lccode`~=`\^^M
   \lowercase{\endgroup\def~}{\par\leavevmode}}
  {\ignorespacesafterend}

%%%%%%%%%%%%%%%%%%%%%%%%%%%%%%%%%%%%%%%%%%%%%%%%%%%%%%%%%%%%%%%%%%%%%%%%%%%%%%%

\begin{document}
\title{Beowulf Callback Script}
\date{13/10/2016}
\author{
Daniel Clay \\ 
}
\maketitle

%%%%%%%%%%%%%%%%%%%%%%%%%%%%%%%%%%%%%%%%%%%%%%%%%%%%%%%%%%%%%%%%%%%%%%%%%%%%%%%
\section{Scripts}
%%%%%%%%%%%%%%%%%%%%%%%%%%%%%%%%%%%%%%%%%%%%%%%%%%%%%%%%%%%%%%%%%%%%%%%%%%%%%%%

\linewise{

\subsection{Passage 1}

\centerline{In this passage, Beowulf describes how he beat Grendel.}
\centerline{ He should sound proud of his deeds, but not boastful.}
\centerline{ This is an opportunity to show an ability to tell a story.}

\textbf{Beowulf} We willingly undertook this test of courage,
risked a match with the might of the stranger,
and performed it all. I would prefer, though,
that you had rather seen the rest of him here,
the whole length of him, lying here dead.
I had meant to catch him, clamp him down
with a cruel lock to his last resting-place;
with my hands upon him, I would have him soon
in the throes of death – unless he disappeared!
But I had not a good enough grip to prevent
his getting away, when God did not wish it;
the fiend in his flight was far too violent,
my life’s enemy. But he left his hand
behind him here, so as to have his life,
and his arm and shoulder. And all for nothing:
it bought him no respite, wretched creature.

He lives no longer, laden with sins,
to plague mankind: pain has set
heavy hands on him, and hasped about him
fatal fetters. 

\subsection{Passage 2}

\centerline{As Beowulf prepares to confront Grendel's Mother, he is given a sword
by Unferth (called \textit{Hrunting}).}
\centerline{He then addresses Hrothgar before diving into the lake.}

\centerline{\textit{Having strapped on the new sword, Beowulf turns to Hrothgar.}}

\textbf{Beowulf} I am eager to begin, great son of Healfdene.
Remember well, then, my wise lord,
provider of gold, what we agreed once before,
that if in your service it should so happen
that I am sundered from life, that you would assume the place
of a father towards me when I was gone.
Now extend your protection to the troop of my companions,
my young fellows, if the fight should take me;
convey also the gifts that you have granted to me,
beloved Hrothgar, to my lord Hygelac.
For on seeing this gold, the Geat chieftain,
Hrethel’s son, will perceive from its value
that I had met with magnificent patronage
from a giver of jewels, and that I had joy of him.

Let Unferth have the blade that I inherited
this wave-patterned sword
of rare hardness. With Hrunting shall I
achieve this deed – or death shall take me!’

\subsection{Passage 3}

\centerline{Before the Geats leave for home, Beowulf addresses Hrothgar}
\centerline{ and thanks him for his hosting, before offering him aid if required.}

\textbf{Beowulf} We wish now to say, seafarers who
are come from afar, how keenly we desire
to return again to Hygelac. Here we were rightly,
royally, treated; you have entertained us well.

If I can ever on this earth earn of you,
O lord of men, more of your love
than I have so far done, by deeds of war,
I shall at once be ready. If ever I hear
that the neighbouring tribes intend your harm,
as those who hate you have done in the past,
I’ll bring a thousand thegns and heroes
here to help you. As for Hygelac, I know
that the Lord of the Geats, Guide of his flock,
young though he is, will yield his support
both in words and deeds so I may do you honour
and bring you a grove of grey-tipped spears
and my strength in aid when you are short of men.

\subsection{Passage 4}

\centerline{Before facing the dragon, Beowulf once again gives a speech to his companions.}
\centerline{ This speech provides an interesting contrast to his earlier speeches:}
\centerline{old and young; enthusiastic and aged, and in the fights; the earlier fights sought out,} 
\centerline{this one brought upon him. Not that Beowulf is still a hero; despite everything}
\centerline{he still faces the challenge head on.}


\textbf{Beowulf} Many were the struggles I survived in youth
in times of danger; I do not forget them.
When that open-handed lord beloved by the people
received me from my father I was seven years old:
King Hrethel kept and fostered me,
gave me treasure and table-room, true to our kinship.
All his life he had as little hatred for me,
a warrior in hall, as he had for a son,
Herebeald, or Hathkin, or Hygelac my own lord.

I had the fortune in battle, by my bright sword,
to make return to Hygelac for the treasures he had given me.
He had granted me land, land to enjoy
and leave to my heirs. Little need was there
that Hygelac should go to the Gifthas or the Spear-Danes
or seek out ever in the Swedish kingdom
a weaker champion, and chaffer for his services.
I was always before him in the footing host,
by myself in the front.

Now shall hard edge,
hand and blade, do battle for the hoard!

Battles in plenty
I ventured in youth; and I shall venture this feud
and again achieve glory, the guardian of my people,
old though I am, if this evil destroyer
dares to come out of his earthen hall.’

\centerline{\textit{He addresses his companions}}

\textbf{Beowulf} I would choose not to take
any weapon to this worm, if I well knew
of some other fashion fitting to my boast
of grappling with this monster, as with Grendel before.
But as I must expect here the hot war-breath
of venom and fire, for this reason I have
my shield and armour. From the keeper of the barrow
I shall not flee one foot; but further than that
shall be worked out at the wall as our fate is given us
by the Creator of men. My mood is strong;
I forgo further words against the winged fighter.

Men in armour! Your mail-shirts protect you:
await on the barrow the one of us two
who shall be better able to bear his wounds
after this onslaught. This affair is not for you,
nor is it measured to any man but myself alone
to match strength with this monstrous being,
attempt this deed. By daring will I
win this gold; war otherwise
shall take your king, terrible life’s-bane!

}

\end{document}