\documentclass[a4paper]{article}

% Table of contents depth (currently unused)
\setcounter{tocdepth}{3}

% Section numbering depth (zero for no numbering)
\setcounter{secnumdepth}{0}

% latex package inclusions here
%\usepackage{fullpage}
\usepackage{hyperref}
\usepackage{tabulary}
%\usepackage{amsthm}

% set up BNF generator
%\usepackage{syntax}
%\setlength{\grammarparsep}{10pt plus 1pt minus 1pt}
%\setlength{\grammarindent}{10em} 

% set up source code inclusion
\usepackage{listings}
\lstset{
  tabsize=2,
  basicstyle = \ttfamily\small,
  columns=fullflexible
}
% Usage for the above like so:
% \begin{lstlisting}
%   CODE CODE CODE
% \end{lstlisting}

% in-line code styling (same style as listing)
\newcommand{\shell}[1]{\lstinline{#1}}

% Line and paragraph spacing
\newenvironment{linewise}
  {\parindent=0pt
   \obeyspaces\obeylines
   \begingroup\lccode`~=`\^^M
   \lowercase{\endgroup\def~}{\par\leavevmode}}
  {\ignorespacesafterend}

%%%%%%%%%%%%%%%%%%%%%%%%%%%%%%%%%%%%%%%%%%%%%%%%%%%%%%%%%%%%%%%%%%%%%%%%%%%%%%%

\begin{document}
\title{Beowulf Callback Script}
\date{13/10/2016}
\author{
Daniel Clay \\ 
}
\maketitle

%%%%%%%%%%%%%%%%%%%%%%%%%%%%%%%%%%%%%%%%%%%%%%%%%%%%%%%%%%%%%%%%%%%%%%%%%%%%%%%
\section{Scripts}
%%%%%%%%%%%%%%%%%%%%%%%%%%%%%%%%%%%%%%%%%%%%%%%%%%%%%%%%%%%%%%%%%%%%%%%%%%%%%%%

\subsection{Beowulf}%%%%%%%%%%%%%%%%%%%%%%%%%%%%%%%%%%%%%%%%%%%%%%%%%%%%%%%%%%%%%%

\linewise{

\centerline{In this passage, Beowulf describes how he beat Grendel. He should sound
proud of his deeds, but not boastful. This is an opportunity to show an ability
to tell a story.}

\textbf{Beowulf} We willingly undertook this test of courage,
risked a match with the might of the stranger,
and performed it all. I would prefer, though,
that you had rather seen the rest of him here,
the whole length of him, lying here dead.
I had meant to catch him, clamp him down
with a cruel lock to his last resting-place;
with my hands upon him, I would have him soon
in the throes of death – unless he disappeared!
But I had not a good enough grip to prevent
his getting away, when God did not wish it;
the fiend in his flight was far too violent,
my life’s enemy. But he left his hand
behind him here, so as to have his life,
and his arm and shoulder. And all for nothing:
it bought him no respite, wretched creature.

He lives no longer, laden with sins,
to plague mankind: pain has set
heavy hands on him, and hasped about him
fatal fetters. 

\centerline{As Beowulf prepares to confront Grendel's Mother, he is given a sword
by Unferth (called \textit{Hrunting}). He then addresses Hrothgar before diving into the lake.}

\centerline{\textit{Having strapped on the new sword, Beowulf turns to Hrothgar.}}

\textbf{Beowulf} I am eager to begin, great son of Healfdene.
Remember well, then, my wise lord,
provider of gold, what we agreed once before,
that if in your service it should so happen
that I am sundered from life, that you would assume the place
of a father towards me when I was gone.
Now extend your protection to the troop of my companions,
my young fellows, if the fight should take me;
convey also the gifts that you have granted to me,
beloved Hrothgar, to my lord Hygelac.
For on seeing this gold, the Geat chieftain,
Hrethel’s son, will perceive from its value
that I had met with magnificent patronage
from a giver of jewels, and that I had joy of him.

Let Unferth have the blade that I inherited
this wave-patterned sword
of rare hardness. With Hrunting shall I
achieve this deed – or death shall take me!’

\centerline{Before the Geats leave for home, Beowulf addresses Hrothgar and thanks him
for his hosting, before offering him aid if required.}

\textbf{Beowulf} We wish now to say, seafarers who
are come from afar, how keenly we desire
to return again to Hygelac. Here we were rightly,
royally, treated; you have entertained us well.

If I can ever on this earth earn of you,
O lord of men, more of your love
than I have so far done, by deeds of war,
I shall at once be ready. If ever I hear
that the neighbouring tribes intend your harm,
as those who hate you have done in the past,
I’ll bring a thousand thegns and heroes
here to help you. As for Hygelac, I know
that the Lord of the Geats, Guide of his flock,
young though he is, will yield his support
both in words and deeds so I may do you honour
and bring you a grove of grey-tipped spears
and my strength in aid when you are short of men.

\centerline{Before facing the dragon, Beowulf once again gives a speech to his
companions. This speech provides an interesting contrast to his earlier speeches:
old and young; enthusiastic and aged, and in the fights; the earlier fights sought out, 
this one brought upon him. Not that Beowulf is still a hero; despite everything
he still faces the challenge head on.}


\textbf{Beowulf} Many were the struggles I survived in youth
in times of danger; I do not forget them.
When that open-handed lord beloved by the people
received me from my father I was seven years old:
King Hrethel kept and fostered me,
gave me treasure and table-room, true to our kinship.
All his life he had as little hatred for me,
a warrior in hall, as he had for a son,
Herebeald, or Hathkin, or Hygelac my own lord.

I had the fortune in battle, by my bright sword,
to make return to Hygelac for the treasures he had given me.
He had granted me land, land to enjoy
and leave to my heirs. Little need was there
that Hygelac should go to the Gifthas or the Spear-Danes
or seek out ever in the Swedish kingdom
a weaker champion, and chaffer for his services.
I was always before him in the footing host,
by myself in the front.

Now shall hard edge,
hand and blade, do battle for the hoard!

Battles in plenty
I ventured in youth; and I shall venture this feud
and again achieve glory, the guardian of my people,
old though I am, if this evil destroyer
dares to come out of his earthen hall.’

\centerline{\textit{He addresses his companions}}

\textbf{Beowulf} I would choose not to take
any weapon to this worm, if I well knew
of some other fashion fitting to my boast
of grappling with this monster, as with Grendel before.
But as I must expect here the hot war-breath
of venom and fire, for this reason I have
my shield and armour. From the keeper of the barrow
I shall not flee one foot; but further than that
shall be worked out at the wall as our fate is given us
by the Creator of men. My mood is strong;
I forgo further words against the winged fighter.

Men in armour! Your mail-shirts protect you:
await on the barrow the one of us two
who shall be better able to bear his wounds
after this onslaught. This affair is not for you,
nor is it measured to any man but myself alone
to match strength with this monstrous being,
attempt this deed. By daring will I
win this gold; war otherwise
shall take your king, terrible life’s-bane!

}

\subsection{Bards}%%%%%%%%%%%%%%%%%%%%%%%%%%%%%%%%%%%%%%%%%%%%%%%%%%%%

\linewise{

\centerline{This short excerpt describes Grendel's approach to Heorot. While much of the Bards' performances are
light hearted, this passage needs to be tense and build a sense of foreboding.}

\centerline{\textit{Grendel approaches the hall slowly. The lighting is colder now, as if lit my moonlight.}}

\textbf{The Bards} Gliding through the shadows came
the walker in the night; the warriors slept
whose task was to hold the horned building,
all except one. He awaited, heart swelling
with anger against his foe, the ordeal of battle.

Down off the moorlands’ misting fells came
Grendel stalking; God’s brand was on him.
The spoiler meant to snatch away
from the high hall some of human race.

That was not the first visit
he had paid to the hall of Hrothgar the Dane:
he never before and never after
harder luck nor hall-guards found.

\centerline{This passage describes how Beowulf swims down through the lake to confront Grendel's Mother.
It's a nice opportunity for a bit of inter-bard interaction and some lovely poetry.}

\textbf{The Bards} After these words the Weather-Geat prince
dived into the Mere – he did not care
to wait for an answer – and the waves closed over
the daring man. It was a day’s space almost
before he could glimpse ground at the bottom.
The grim and greedy guardian of the flood,
keeping her hungry hundred-season watch,
discovered at once that one from above,
a human, had sounded the home of the monsters.

She felt for the man and fastened upon him
her terrible hooks; but no harm came thereby
to the body within – the armour so ringed him
that she could not drive her dire fingers
through the mesh of the mail-shirt masking his limbs.
When she came to the bottom she bore him to her lair,
the mere-wolf, pinioning the mail-clad prince.

Then the man found
that he was in some enemy hall
where there was no water to weigh upon him
and the power of the flood could not pluck him away,
sheltered by its roof: a shining light he saw,
a bright fire blazing clearly.

\centerline{The following passage is really the happiest point of the play.
Beowulf has vanquished his two enemies and is now returning in triumph to Geatland.
As such, this passage should be lively and gives a great opportunity for the bards
to have some fun with the narrative as the interval draws closer.}

\textbf{The Bards} Beowulf went from him,
trod the green earth, a gold-resplendent warrior
rejoicing in his rings. Riding at anchor
the strayer of ocean stayed for her master.
Chiefly the talk returned as they walked
to Hrothgar’s giving. He was a king
blameless in all things, until old age at last,
that brings down so many, removed his proud strength.

They came then to the sea-flood, the spirited band
of warrior youth, wearing the ring-meshed
coat of mail.
The wide sea-boat with its soaring prow
was loaded at the beach there with battle-raiment,
with horses and arms. High rose the mast
above the lord Hrothgar’s hoard of gifts.

Out moved the boat then
to divide the deep water, left Denmark behind.
A special sea-dress, a sail, was hoisted
and belayed to the mast. The beams spoke.
The wind did not hinder the wave-skimming ship
as it ran through the seas, but the sea-going craft
with foam at its throat, furled back the waves,
her ring-bound prow planing the waters
till they caught sight of the cliffs of the Geats
and headlands they knew. The hull drove ahead,
urged by the breeze, and beached on the shore.

The harbour-guard was waiting at the water’s edge;
his eye had been scouring the stretches of the flood
in a long look-out for these loved men.
Now he moored the broad-ribbed boat in the sand,
held fast with hawsers, so no heft of the waves
should drive away again those darling timbers.

He had the heroes’ hoard brought ashore,
their gold-plated armour. To go to their lord
was now but a step, to see again Hygelac
the son of Hrethel, at his home where he dwells
himself with his hero-band, hard by the sea-wall.

That was a handsome hall there. And high within it sat
a king of great courage. His consort was young,
but wise and discreet for one who had lived
so few years at court; the queen’s name was Hygd,
Hareth’s daughter. When she dealt out treasure
to the Geat nation, the gifts were generous,
there was nothing narrowly done.

The war-man himself came walking along
by the broad foreshores with his band of picked men,
trod the sea-beach. From the south blazed
the sun, the world’s candle. They carried themselves forward,
stepping on eagerly to the stronghold where
Ongentheow’s conqueror, the earls’ defender,
the warlike young king, was well-known for his
giving of neck-rings. The news of Beowulf’s
return was rapidly told to Hygelac
– that the shield of warriors, his own shoulder-companion,
had walked alive within the gates,
unscathed from the combat, and was coming to the hall.

\centerline {This final passage is the closing of the play, taking the form of a narration of
Beowulf's funeral arrangements along with a brief dirge. The mood is solemn. As the narrative
unfolds, the actors will construct and set alight a funeral pyre.}

\centerline{\textit{The people construct a funeral pyre and place Beowulf on top of it.}}

\textbf{The Bards} The Geat race then reared up for him
a funeral pyre. It was not a petty mound,
but shining mail-coats and shields of war
and helmets hung upon it, as he had desired.
Then the heroes, lamenting, laid out in the middle
their great chief, their cherished lord.

On top of the mound the men then kindled
the biggest of funeral-fires. Black wood-smoke
arose from the blaze, and the roaring of flames
mingled with weeping. The winds lay still
as the heat at the fire’s heart consumed
the house of bone. And in heavy mood
they uttered their sorrow at the slaughter of their lord.

\centerline{\textit{The fire is lit. Smoke begins to rise.}}

A woman of the Geats in grief sang out
the lament for his death. Loudly she sang,
her hair bound up, the burden of her fear
that evil days were destined her
– troops cut down, terror of armies,
bondage, humiliation. Heaven swallowed the smoke.

Then the Storm-Geat nation constructed for him
a stronghold on the headland, so high and broad
that seafarers might see it from afar.
The beacon to that battle-reckless man
they made in ten days. What remained from the fire
they cast a wall around, of workmanship
as fine as their wisest men could frame for it.

\centerline{\textit{The fire grows.}}

They placed in the tomb both the torques and the jewels,
all the magnificence that the men had earlier
taken from the hoard in hostile mood.
They left the earls’ wealth in the earth’s keeping,
the gold in the dirt. It dwells there yet,
of no more use to men than in ages before.

Then the warriors rode around the barrow,
twelve of them in all, athelings’ sons.
They recited a dirge to declare their grief,
spoke of the man, mourned their king.

They praised his manhood and the prowess of his hands,
they raised his name; it is right a man
should be lavish in honouring his lord and friend,
should love him in his heart when the leading-forth
from the house of flesh befalls him at last.

This was the manner of the mourning of the men of the Geats,
sharers in the feast, at the fall of their lord:
they said that he was of all the world’s kings
the gentlest of men, and the most gracious,
the kindest to his people, the keenest for fame.

}

\end{document}