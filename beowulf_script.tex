\documentclass[a4paper]{article}

% Table of contents depth (currently unused)
\setcounter{tocdepth}{3}

% Section numbering depth (zero for no numbering)
\setcounter{secnumdepth}{0}

% latex package inclusions here
%\usepackage{fullpage}
\usepackage{hyperref}
\usepackage{tabulary}
%\usepackage{amsthm}

% set up BNF generator
%\usepackage{syntax}
%\setlength{\grammarparsep}{10pt plus 1pt minus 1pt}
%\setlength{\grammarindent}{10em} 

% set up source code inclusion
\usepackage{listings}
\lstset{
  tabsize=2,
  basicstyle = \ttfamily\small,
  columns=fullflexible
}
% Usage for the above like so:
% \begin{lstlisting}
%   CODE CODE CODE
% \end{lstlisting}

% in-line code styling (same style as listing)
\newcommand{\shell}[1]{\lstinline{#1}}

% Line and paragraph spacing
\newenvironment{linewise}
  {\parindent=0pt
   \obeyspaces\obeylines
   \begingroup\lccode`~=`\^^M
   \lowercase{\endgroup\def~}{\par\leavevmode}}
  {\ignorespacesafterend}

%%%%%%%%%%%%%%%%%%%%%%%%%%%%%%%%%%%%%%%%%%%%%%%%%%%%%%%%%%%%%%%%%%%%%%%%%%%%%%%

\begin{document}
\title{Beowulf Script}
\date{2016}
\author{
Daniel Clay \\ 
}
\maketitle

%%%%%%%%%%%%%%%%%%%%%%%%%%%%%%%%%%%%%%%%%%%%%%%%%%%%%%%%%%%%%%%%%%%%%%%%%%%%%%%
\section{Introduction}
%%%%%%%%%%%%%%%%%%%%%%%%%%%%%%%%%%%%%%%%%%%%%%%%%%%%%%%%%%%%%%%%%%%%%%%%%%%%%%%

\subsection{Cast}%%%%%%%%%%%%%%%%%%%%%%%%%%%%%%%%%%%%%%%%%%%%%%%%%

In order of appearance:
\begin{description}
    \item[The Bards] Narrators
    \item[Hrothgar] King of the Spear-Danes (Scyldings)*
    \item[Grendel] An evil monster, descendant of Cain*
    \item[Beowulf] A hero of the Geatish people
    \item[Beowulf's companions] 6 Geatish thegns*
    \item[Wulfgar] Hrothgar's Counsellor*
    \item[Unferth] Hrothgar's spokesman*
    \item[Wealhtheow] Queen of the Spear-Danes*
    \item[Grendel's Mother] A monstrous ogress*
    \item[Ashhere] Hrothgar's Counsellor*
    \item[Hygelac] King of the Geats**
    \item[Hygd] Queen of the Geats**
    \item[The Dragon] A dragon, awakened from it's slumber**
    \item[Wiglaf] Beowulf's faithful companion**
    \item[Messenger] A Geat, who brings word of Beowulf's death to his court**
    \item[Beowulf's companions] 6 Geatish thegns**
\end{description}

Of the cast, most of Beowulf's companions can be omitted as required, and those
present at the court of Hrothgar (marked with a  *) can be doubled up with those present at the court
of Hygelac (marked with a **; characters who cannot be doubled up are unmarked).

Likewise, additional characters may be present at both courts - councillors, warriors and servants.

Most of the minor roles can be played by men or women, as can Grendel, Grendel's
Mother, the Dragon (possibly a two person costume) and The Bards.

In total, the play can be performed by anywhere upwards of 10 people, with an ideal number being around 14-20.

\subsection{Staging}%%%%%%%%%%%%%%%%%%%%%%%%%%%%%%%%%%%%%%%%%%%%%%%%%

This play works best when some of the character of the mead-hall, for which the
original work was composed, is retained. Staging should be minimal and as far as
possible abstracted away. No scene changes should be required, but can be suggested 
by lighting or simple changes of set.

Lighting by open flame is preferred to lighting by electric light.

\subsection{Structure of the play}%%%%%%%%%%%%%%%%%%%%%%%%%%%%%%%%%

Despite the length of the original poem, which runs to over 3000 lines, the play
should be able to be performed straight through, without an interval.

If an interval is desired, it should happen after Act II.

The poem itself is loosely organised in three acts, characterised by the three 
combats Beowulf takes part in; that with Grendel, with Grendel's Mother, and with the Dragon.

\newpage

%%%%%%%%%%%%%%%%%%%%%%%%%%%%%%%%%%%%%%%%%%%%%%%%%%%%%%%%%%%%%%%%%%%%%%%%%%%%%%%
\section{Script}
%%%%%%%%%%%%%%%%%%%%%%%%%%%%%%%%%%%%%%%%%%%%%%%%%%%%%%%%%%%%%%%%%%%%%%%%%%%%%%%

\subsection{Act 1}%%%%%%%%%%%%%%%%%%%%%%%%%%%%%%%%%%%%%%%%%%%%%%%%

\linewise{

\centerline{\textbf{Scene 1 - Introduction}}
\centerline{\textit{A mead hall, somewhere in 6th Century Denmark.}}
\centerline{\textit{\textbf{The Bards} address the audience as if they were guests in the mead hall.}}

\textbf{The Bards} Listen!
We have heard of the thriving of the throne of Denmark,
how the folk-kings flourished in former days,
how those royal athelings earned that glory.

Was it not Scyld Shefing that shook the halls?
Took mead-benches? Taught encroaching
foes to fear him? Who, found in childhood,
lacked clothing? Yet he lived and prospered,
grew in strength and stature under the heavens
until the clans settled in the sea-coasts neighbouring
over the whale-road must all obey him
and give tribute. He was a good king!

At the hour shaped for him Scyld departed,
the hero crossed into the keeping of his Lord.

Then for a long space there lodged in the stronghold
Beow the Dane, dear king of his people,
when late was born to him
the lord Healfdene, lifelong the ruler
and war-feared patriarch of the proud Scyldings.

He next fathered four children
that leapt into the world, this leader of armies,
Heorogar and Hrothgar and Halga the Good and Ursula.

Then to Hrothgar was granted glory in battle,
mastery of the field; so friends and kinsmen
gladly obeyed him, and his band increased
to a great company. It came into his mind
that he would command the construction
of a huge mead-hall, a house greater
than men on earth ever had heard of,
and share the gifts God had bestowed on him
upon its floor with folk young and old.

\newpage
Far and wide the work was given out
in many a tribe over middle earth,
the making of the mead-hall. And, as men reckon,
the day of readiness dawned very soon
for this greatest of houses. Heorot he named it
whose word ruled a wide empire.

He made good his boast, gave out rings,
arm-bands at the banquet. Boldly the hall reared
its arched gables; unkindled the torch-flame
that turned it to ashes. The time was not yet
when the blood-feud should bring out again
sword-hatred in sworn kindred.

It was with pain that the powerful spirit
dwelling in darkness endured that time,
hearing daily the hall filled with loud amusement.
\textit{Grendel} they called this cruel spirit,
the fell and fen his fastness was, the march his haunt.

With the coming of night came Grendel also,
seeking the great house and how the Ring-Danes
held their hall after the horn had gone round.
He found in Heorot the force of nobles
slept after supper, sorrow forgotten,
the condition of men. Maddening with rage,
he struck quickly, creature of evil:
grim and greedy, he grasped on their pallets
thirty warriors, and away he was out of there,
thrilled with his catch: he carried off homeward
his glut of slaughter, sought his own halls.

Sorrow forgotten, the condition of men. 
Maddening with rage, this creature of evil 
struck quickly.

As the day broke, with the dawn’s light
Grendel’s outrage was openly to be seen:
night’s table-laughter turned to morning’s
lamentation. Lord Hrothgar
sat silent then, the glorious king mourned, 
grieving for his thegns as they read the 
traces of a terrible foe,
a cursed fiend. That was too cruel a feud,
too long, too hard!

\newpage
Nor did he let them rest
but the next night brought new horrors,
did more murder, manslaughter and outrage
and shrank not from it: he was too set on these things.

So Grendel became ruler; against right he fought,
one against all. Empty then stood Heorot,
the best of houses, and for no brief space.

For twelve long winters torment sat
on the Spear-Danes. A great grief it was for the
Guardian of the Scyldings, crushing to his
spirit. The council lords sat there daily to devise some plan, 
what might be best for the brave-hearted Danes 
to contrive against these terror-raids.

This was heard of at his home by one of Hygelac’s followers,
a good man among the Geats, Noble and strong,
he was above all men that trod the earth at that time; 
build and blood matched.

The prince had already picked his men
from the folk’s flower, the fiercest among them
that might be found. With six men
he sought sound-wood; and led them right down 
to the land's edge.

He bade a seaworthy 
wave-cutter to be fitted out for him; the warrior king, 
Hrothgar, he would seek, he said.

Away she went over a wavy ocean,
boat like a bird, breaking seas,
wind-whetted, white-throated,
till the curved prow had ploughed so far
that they saw land loom on the skyline,
then the shimmer of cliffs, sheer fells behind,
reaching capes.

The crossing was at an end;
closed the wake. Weather-Geats
stood on land. God they thanked
for the smooth going over the salt trails.

There was stone paving on the path that brought
the war-band on its way. The war-coats shone
and the war-banners cried out the arrival of the prince 
of the Geats, that hero to his people: Beowulf!

\newpage
\centerline{\textbf{Scene 2 - Beowulf's Arrival at Heorot}}
\centerline{\textit{Beowulf \& Companions approach the hall.}}
\centerline{\textit{Loud knocking on the door, which then swings open}}
\centerline{\textit{They enter. Wulfgar stands and walks over to greet/challenge them.}}

\textbf{Wulfgar} From whence do you bring these embellished shields, 
grey mail-shirts, masked helmets, 
this stack of spears? I am spokesman here,
herald to Hrothgar. I have not seen
a body of strangers bear themselves more proudly.

It is not exile but adventure, I think,
boldness of spirit, that brings you to Hrothgar.

Nonetheless, I’ll have your names now
and the names of your fathers; or further you shall not go
as strangers in the court of Hrothgar.

\textbf{Beowulf} At Hygelac's table we are sharers in the banquet.
Beowulf is my name.
I shall gladly set out to the son of Healfdene,
most famous of kings, the cause of my journey,
lay it before your lord, if he will allow us kindly,
to greet in person his most gracious self.

\textbf{Wulfgar} The Master of the Danes,
Lord of the Scyldings, shall learn of your request.

\centerline{\textit{Wulfgar walks over to Hrothgar to tell him who the strangers are.}}

\textbf{Wulfgar} Men have come here from the country of the Geats,
borne from afar over tha back of the sea;
these battle-companions call the man who leads them Beowulf.

The boon they ask is, my lord, that they may converse with you. 
Do not, kind Hrothgar, refuse them audience in the answer you vouchsafe.
Their war-gear would clearly bespeak them of Earl's rank. 
Indeed, the leader who guided them here seems of great account.

\textbf{Hrothgar} I knew him when he was a child!
It was to his old father, Edgetheow, that
Hrethel the Geat gave in marriage
his one daughter. Well does the son
now pay this call on a proven ally!

The seafarers used to say, I remember,
who took our gifts to the Geat people
in token of friendship – that this fighting man
in his hand’s grasp had the strength
of thirty other men. I am thinking that
the Holy God, as a grace to us
Danes in the West, has directed him here
against Grendel’s oppression. This good man shall be
offered treasures in return for his courage.

Waste no time now but tell them to come in
that they may see this company seated together.
Make sure to say that they are most welcome
to the people of the Danes.

\centerline{\textit{Wulfgar walks back to Beowulf and his companions, who have been waiting silently.}}

\textbf{Wulfgar} The Master of Battles bids me announce,
the Lord of the North Danes, that he knows your ancestry;
I am to tell you all, determined venturers
over the seas, that you are sure of welcome.
You may go in now in your gear of battle,
set eyes on Hrothgar, helmed as you are.

But battle-shafts and shields of linden wood
may here await your words' outcome.

\centerline{\textit{The Geats hand their weapons to two of their number, who remain to guard them.}}
\centerline{\textit{The rest follow Beowulf towards the king, now unarmed but still in armour.}}

\textbf{Beowulf} Health to Hrothgar! I am Hygelac’s kinsman
and serve in his fellowship. Fame-winning deeds
have come early to my hands. The affair of Grendel
has been made known to me on my native turf.

Whereupon it was urged by the ablest men
among our people, men proved in counsel,
that I should seek you out, most sovereign Hrothgar.

These men knew well the weight of my hands.
Had they not seen me come home from fights
where I had bound five Giants?
Had I not crushed on the wave
sea-serpents by night in narrow struggle,
broken the beasts? And shall I not try
a single match with this monster Grendel,
a trial against this troll?

To you I now
put one request, Royal Scylding,
Shield of the South Danes, one sole favour
that you’ll not deny me, dear lord of your people,
now that I have come thus far, Fastness of Warriors;
that I alone may be allowed, with my loyal and determined
band of companions, to cleanse your hall, Heorot.

As I am informed that this unlovely one
is careless enough to carry no weapon,
so that my lord Hygelac, my leader in war,
may take joy in me, I abjure utterly
the bearing of sword or shielding yellow
board in this battle! With bare hands shall I
grapple with the fiend, fight to the death here,
hater and hated! He who is chosen
shall deliver himself to the Lord’s judgement.
Fate will take it's course!

\textbf{Hrothgar} So it is to fight in our defence, my friend Beowulf,
and as an office of kindness that you have come to us here!

It is a sorrow in spirit for me to say to any man
what the hatred of Grendel has brought me to in Heorot, 
what humiliation, what harrowing pain. 
My hall-companions, my war-band, are dwindled; 
fate has swept them into the power of Grendel.

Yet sit now to the banquet, where you may soon attend,
should the mood so take you, some tale of victory.’

\centerline{\textit{A bench is cleared for the Geats, and they all sit.}}
\centerline{\textit{Mead is brought around and laughter is heard. Talking amongst the cast.}}

\newpage
\centerline{\textbf{Scene 3 - Feasting and Bragging}}
\centerline{\textit{After some time, an increasingly irritated Unferth stands up.}}

\textbf{Unferth} Is this the Beowulf of Breca's swimming match, 
who strove against him on the stretched ocean,
when for pride the pair of you proved the seas
and for a trite boast entrusted your lives
to the deep waters? A sorry contest!

Your arms embraced the ocean’s streams, in the water’s power
you laboured seven nights: and then you lost your swimming-match!

I see little hope of a happier outcome if you propose awaiting
Grendel all night, on his own ground, unarmed.

\textbf{Beowulf} I thank my friend Unferth, who unlocks us this tale
of Breca's bragged exploit; 
the beer lends eloquence to his tongue.
But the truth is as I’ve said:
I had more sea-strength, outstaying Breca’s,
and endured underwater a much worse struggle.

Hard in our right hands we held each a sword
as we went through the sea, so to keep off
the whales from us. If he whitened the ocean,
no wider appeared the water between us.
He could not away from me; nor would I from him.
Thus stroke for stroke we stitched the ocean
five nights and days, drawn apart then
by cold storm on the cauldron of waters;
under lowering night the northern wind
fell on us in warspite: the waves were rough!

The unfriendliness was then aroused of the fishes of the deep.
Against sea-beasts my body-armour,
hand-linked and hammered, helped me then,
this forge-knit battleshirt bright with gold,
decking my breast. Down to the bottom
I was plucked in rage by this reptile-fish,
pinned in his grip. But I got the chance
to thrust once at the ugly creature
with my weapon’s point: war took off then
the mighty monster; mine was the hand did it.

\newpage
It was my part then to put to the sword
nine sea-monsters, in the severest fight
by night I have heard of under heaven’s vault;
a man more sorely pressed the seas never held.
I came with my life from the compass of my foes,
but tired from the struggle. The tide bore me
away on its currents to the coasts of Norway,
whelms of water.

No whisper has yet reached me
of sword-ambushes survived, nor such scathing perils
in connection with your name! Never has Breca,
nor you Unferth either, in open battle-play
framed such a deed of daring with your
shining swords – small as my action was.
You have killed only kindred, kept your blade
for those closest in blood; you’re a clever man, Unferth,
but you’ll endure hell’s damnation for that.

It speaks for itself, my son of Edgelaf,
that Grendel had never grown such a terror,
this demon had never dealt your lord
such havoc in Heorot, had your heart’s intention
been so grim for battle as you give us to believe.

He’s learnt there’s in fact not the least need
excessively to respect the spite of this people,
the scathing steel-thresh of the Scylding nation.
He spares not a single sprig of your Danes
in extorting his tribute, but treats himself proud,
butchering and dispatching, and expects no resistance
from the spear-wielding Scyldings.

I’ll show him Geatish
strength and stubbornness shortly enough now,
a lesson in war. He who wishes shall go then
blithe to the banquet when the breaking light
of another day shall dawn for men
and the sun shine glorious in the southern sky.

\newpage
\centerline{\textit{The partying resumes, and Wealhtheow enters with the ceremonial mead-cup.}}
\centerline{\textit{She passes it around, starting with Hrothgar, before coming last to Beowulf.}}

\textbf{Wealhtheow} Hail, son of Edgetheow,
honoured guest of the Spear-Danes.

I thank the Almighty Lord, our Maker
to see so mighty a man before us
to be counted on for aid in our troubles.

\centerline{\textit{Beowulf takes the cup and drinks.}}

\textbf{Beowulf} This was my determination in taking to the ocean,
benched in the ship among my band of fellows,
that I should once and for all accomplish the wishes
of your adopted people, or pass to the slaughter,
viced in my foe’s grip. This vow I shall accomplish,
a deed worthy of an earl; decided otherwise
here in this mead-hall to meet my ending-day!

\centerline{\textit{Wealhtheow retrieves the cup and moves back to sit beside Hrothgar.}}

\textbf{The Bards} Then at last Heorot heard once more
words of courage, the carousing of a people
singing their victories; till the son of Healfdene
desired at length to leave the feast,
be away to his night’s rest; aware of the monster
brooding his attack on the tall-gabled hall
from the time they had seen the sun’s lightness
to the time when darkness drowns everything
and under its shadow-cover shapes do glide
dark beneath the clouds. 
The company came to its feet.

\centerline{\textit{All stand. Hrothgar and Beowulf come together to speak.}}

\textbf{Hrothgar} Never since I took up shield and sword
have I at any instance to any man beside,
thus handed over Heorot, as I here do to you.
Have and hold now the house of the Danes!
Bend your mind and your body to this task
and wake against the foe! There’ll be no want of liberality
if you come out alive from this ordeal of courage.

\newpage
\centerline{\textit{Hrothgar and and his people leave. The Geats begin to take off their armour.}}
\centerline{\textit{Mattresses are taken out and the Geats lie down on them. Beowulf addresses the audience.}}

\textbf{Beowulf} I fancy my fighting-strength, my performance in combat,
at least as greatly as Grendel does his;
and therefore I shall not cut short his life
with a slashing sword – too simple a business.
He has not the art to answer me in kind,
hew at my shield, shrewd though he be
at his nasty catches. No, we’ll at night play
without any weapons – if unweaponed he dare
to face me in fight. The Father in His wisdom
shall apportion the honours then, the All-holy Lord,
to whichever side shall seem to Him fit.

\centerline{\textit{The lights fade to darkness}}

\newpage
\centerline{\textbf{Scene 4 - Grendel}}
\centerline{\textit{Grendel approaches the hall slowly. The lighting is colder now, as if lit by moonlight.}}

\textbf{The Bards} Gliding through the shadows came
the walker in the night; the warriors slept
whose task was to hold the horned building,
all except one. He awaited, heart swelling
with anger against his foe, the ordeal of battle.

Down off the moorlands’ misting fells came
Grendel stalking; God’s brand was on him.
The spoiler meant to snatch away
from the high hall some of human race.

That was not the first visit
he had paid to the hall of Hrothgar the Dane:
he never before and never after
harder luck nor hall-guards found.

\centerline{\textit{Grendel smashes through the doors to the hall, ripping them open.}}
\centerline{\textit{He stalks inside and grabs a sleeping warrior, killing him.}}
\centerline{\textit{Grendel moves over towards Beowulf, who stands.}}

\centerline{\textit{Beowulf tackles Grendel}}
\centerline{\textit{The two wrestle for a while, but eventually Beowulf pins Grendel, and takes hold of his arm.}}
\centerline{\textit{The other warriors try to kill Grendel, but their swords are ineffective.}}
\centerline{\textit{Grendel tries to run, but cannot escape Beowulf's grip.}}
\centerline{\textit{Eventually, Grendel's shoulder tears, and his whole arm rips off. Grendel screams.}}
\centerline{\textit{Grendel flees back through the doors, and off stage.}}
\centerline{\textit{Beowulf hangs Grendel's arm above the door, to cheers.}}

\newpage
\centerline{\textbf{Scene 5 - Grendel's Death is Celebrated}}
\centerline{\textit{The light changes back to warm, once again lit by torchlight.}}
\centerline{\textit{Hrothgar and company enter. The scene shifts to a banquet, celebrating Grendel's death.}}
\centerline{\textit{All are amazed at the sight of Grendel's hand. Hrothgar walks over to inspect it.}}

\textbf{Hrothgar} Let swift thanks be given to the Governor of All,
seeing this sight! I have suffered a thousand
spites from Grendel: but God works ever
miracle upon miracle, the Master of Heaven.
Until yesterday I doubted whether
our afflictions would find a remedy
in my lifetime, since this loveliest of halls
stood slaughter-painted, spattered with blood.
For all my counsellors this was a cruel sorrow,
for none of them imagined they could mount a defence
of the Scylding stronghold against such enemies –
warlocks, demons!

But one man has,
by the Lord’s power, performed the thing
that all our thought and arts to this day
had failed to do. 

Beowulf, I now take you
to my bosom as a son, O best of men,
and cherish you in my heart. Hold yourself well
in this new relation! You will lack for nothing
that lies in my gift of the goods of this world:
lesser offices have elicited reward,
we have honoured from our hoard less heroic men,
far weaker in war. But you have well ensured
by the deeds of your hands an undying honour
for your name for ever. May the Almighty Father
yield you always the success that you yesternight enjoyed!

\textbf{Beowulf} We willingly undertook this test of courage,
risked a match with the might of the stranger,
and performed it all. I would prefer, though,
that you had rather seen the rest of him here,
the whole length of him, lying here dead.
I had meant to catch him, clamp him down
with a cruel lock to his last resting-place;
with my hands upon him, I would have him soon
in the throes of death – unless he disappeared!
But I had not a good enough grip to prevent
his getting away, when God did not wish it;
the fiend in his flight was far too violent,
my life’s enemy. 

\newpage
But he left his hand
behind him here, so as to have his life,
and his arm and shoulder. And all for nothing:
it bought him no respite, wretched creature.

He lives no longer, laden with sins,
to plague mankind: pain has set
heavy hands on him, and hasped about him
fatal fetters. 

\centerline{\textit{While they speak, the other characters prepare the feast. Drinks are passed around.}}
\centerline{\textit{Hrothgar returns to his throne, and motions for his spokesman Unferth.}}

\centerline{\textit{Hrothgar speaks quietly to Unferth, who takes some others and goes offstage to get gifts.}}
\centerline{\textit{They come back with a banner, helmet, mail-shirt and sword; all of the highest quality.}}
\centerline{\textit{One by one, Hrothgar gives them to Beowulf, who hands the banner to his companion while he puts the others on.}}

\textbf{Hrothgar} Beloved Beowulf,
great deeds beget great rewards. 
Here I bestow a sign of victory,
this banner worked in gold,
this bright war-coat and helm
this battle-tempered blade.

Take care to use them well.

\centerline{\textit{While the gifts are presented, Unferth collects more gifts for Beowulf's companions.}}
\centerline{\textit{Hrothgar presents the gifts to the companions in turn. He also makes a gift of gold for the dead man.}}

\centerline{\textit{Wealhtheow then brings a great golden cup to Hrothgar.}}

\textbf{Wealhtheow} Accept this cup, my king and lord,
giver of treasure. Let your gaiety be shown,
gold-friend of warriors, and to the Geats speak
in words of friendship, for this well becomes a man.
Be gracious to these Geats, and let the gifts you have had
from near and far, not be forgotten now.

I hear it is your wish to hold this warrior
henceforward as your son. Heorot is cleansed,
the ring-hall bright again: therefore bestow while you may
these blessings liberally, and leave to your kinsmen
the land and its people when your passing is decreed,
your meeting with fate. For may I not count
on my gracious Hrothulf to guard honourably
our young ones here, if you, my lord,
should give over this world earlier than he?

\newpage
I am sure that he will show to our children
answerable kindness, if he keeps in remembrance
all that we have done to indulge and advance him,
the honours we bestowed on him when he was still a child.

\centerline{\textit{Wealhtheow then brings the cup to Beowulf}}

\textbf{Wealhtheow} Be thou hale, my lord.

\centerline{\textit{Unferth returns once more with a necklace and arm-bands. Wealhtheow presents them.}}

\textbf{Wealhtheow} Take pride in this jewel, have joy of this mantle
drawn from our treasuries, most dear Beowulf!
May fortune come with them and may you flourish in your youth!

Already you have so managed that men everywhere
will hold you in honour for all time,
even to the cliffs at the world’s end, washed by the Ocean,
the wind’s range. All the rest of your life
must be happy, prince; and prosperity I wish you too,
abundance of treasure! 

You see how open is each earl here with his neighbour,
temperate of heart, and true to his lord.
The nobles are loyal, the lesser people dutiful;
wine mellows the men to move to my bidding.

\centerline{\textit{She returns to her seat. The celebration continues.}}

\textbf{The Bards} What a banquet that was!
The men drank their wine: the fate they did not know,
destined from of old, the doom that was to fall
on many of the earls there.

\centerline{\textit{Hrothgar exits, and his people clear away the feast.}}
\centerline{\textit{The Geats leave and the other warriors prepare for bed. When finished, the light dims.}}

\textbf{The Bards} The Geats sank into sleep. A savage penalty
one paid for his night’s rest! It was no new thing for that people
since Grendel had occupied the gold-giving hall,
working his evil, until the end came,
death for his misdeeds. It was declared then to men,
and received by every ear, that for all this time
a survivor had been living, an avenger for their foe
and his grim life’s-leaving: Grendel’s Mother herself,
a monstrous ogress, ailing for her loss,
now purposed to set out at last – savage in her grief –
on that wrath-bearing visit of vengeance for her son.

\centerline{\textit{The light fades to darkness}}

\subsection{Act 2}%%%%%%%%%%%%%%%%%%%%%%%%%%%%%%%%%%%%%%%%%%%%%%%%

\centerline{\textbf{Scene 1 - Grendel's Mother Attacks Heorot}}
\centerline{\textit{Lights up dimly, once again cold. Grendel's mother has made her way onstage in the blackout.}}
\centerline{\textit{She creeps slowly around the hall and removes the hand. As she passes one man, he awakes and cries out.}}
\centerline{\textit{Everyone in the hall wakes up and grabs their weapons.}}
\centerline{\textit{Grendel's Mother grabs Ashhere and flees, followed by some of the warriors.}}
\centerline{\textit{Once they are offstage, the remaining warriors sheath their weapons.}}

\newpage
\centerline{\textbf{Scene 2 - The Warriors Take Stock}}
\centerline{\textit{After a short pause, the lights brighten to daylight.}}
\centerline{\textit{Hrothgar and his advisors enter and survey the scene, talking quietly to the remaining warriors.}}
\centerline{\textit{Beowulf and his companions enter in full gear. He goes to speak to Hrothgar}}

\textbf{Beowulf} Kind Hrothgar, has the night been quiet?
Your call was urgent; I came at once. 

\textbf{Hrothgar} Do not ask about our welfare! Woe has returned
to the Danish people with the death of Ashhere,
the elder brother of Yrmenlaf.
He was my closest counsellor, he was keeper of my thoughts,
he stood at my shoulder when we struck for our lives
at the crashing together of companies of foot,
when blows rained on boar-crests. Men of birth and merit
all should be as Ashhere was!

A bloodthirsty monster has murdered him in Heorot,
a wandering demon; whither this terrible one,
glorying in her prey, glad of her meal,
has returned to, I know not. She has taken vengeance
for the previous night, when you put an end to Grendel
with forceful finger-grasp. Revenge is her motive,
and in furthering her son’s feud she has gone far enough,
– or thegns may be found who will think it so;
in their breasts they will grieve for their giver of rings,
bitter at heart. For the hand is stilled
that would openly have granted your every desire.

I have heard it said by subjects of mine
who live in the country, counsellors in this hall,
that they have seen such a pair
of huge wayfarers haunting the moors,
otherworldly ones; and one of them,
so far as they might make it out,
was in woman’s shape; but the shape of a man,
though twisted, trod also the tracks of exile
– save that he was more huge than any human being.

The country people have called him from of old
by the name of Grendel; they know of no father for him,
nor whether there have been such beings before
among the monster-race.

\newpage
Mysterious is the region
they live in – of wolf-fells, wind-picked moors
and treacherous fen-paths: a torrent of water
pours down dark cliffs and plunges into the earth,
an underground flood. It is not far from here,
in terms of miles, that the Mere lies,
overcast with dark, crag-rooted trees
that hang in groves hoary with frost.

Our sole remedy
is to turn again to you. The treacherous country
where that creature of sin is to be sought out
is strange to you as yet: seek then if you dare!
I shall reward the deed, as I did before,
with wealthy gifts of wreathèd ore,
treasures from the hoard, if you return once more.

\textbf{Beowulf} Bear your grief, wise one! It is better for a man
to avenge his friend than to refresh his sorrow.
As we must all expect to leave
our life on this earth, we must earn some renown,
if we can, before death; daring is the thing
for a fighting man to be remembered by.

Let Denmark’s lord arise, and we shall rapidly see then
where this kinswoman of Grendel’s has gone away to!
I can promise you this, that she’ll not protect herself by hiding
in any fold of the field, in any forest of the mountain,
in any dingle of the sea, dive where she will!
For this day, therefore, endure all your woes
with the patience that I may expect of you.

\textbf{Hrothgar} I thank the Almighty Lord
that once again we shall have your aid in our troubles. 
Go now to victory once more,
and triumph over this hell-fiend.

\newpage
\centerline{\textbf{Scene 3 - Journey to the Mere}}
\centerline{\textit{As The Bards narrates, the lighting and set change to indicate that they have travelled from the hall.}}

\textbf{The Bards} A steed with braided mane was bridled then,
a horse for Hrothgar; the hero-patriarch
rode out shining; shieldbearers marched
in troop beside him. The trace of her going
on the woodland paths was plainly to be seen,
stepping onwards; straight across
the fog-bound moor she had fetched away there
the lifeless body of the best man
of all who kept the courts of Hrothgar.
The sons of men then made their way
up steep screes, by scant tracks
where only one might walk, by wall-faced cliffs,
through haunted fens – uninhabitable country.

Going on ahead with a handful of the
keener men to reconnoitre,
Beowulf suddenly saw where some ash-trees
hung above a hoary rock
– a cheerless wood! And the water beneath it
was turbid with blood; bitter distress
was to be endured by the Danes who were there,
a grief for the earls, for every thegn
of the Friends of the Scyldings, when they found there
the head of Ashhere by the edge of the cliff.

The men beheld the blood on the water,
its warm upwellings. The war-horn sang
an eager battle-cry.

\centerline{\textit{Beowulf begins to put on his armour. Once he has his mail-shirt and helmet, Unferth approaches.}}
\centerline{\textit{Unferth unbuckles his sword belt and presents it to Beowulf as a gift.}}

\textbf{Unferth} Forgive my words earlier;
oft wine will lend eloquence beyond a man's prowess.
You have proven himself the better warrior.
This sword is named Hrunting; unique and ancient,
it's edge is iron; annealed in venom and tempered in blood.
In battle it has never failed any hero of my house.
Take it now, and may it serve you well today. 

\textbf{Beowulf} This is indeed a mighty aid, Unferth.
I will take this hilted sword and slay the ogress.

\newpage
\centerline{\textit{Having strapped on the new sword, Beowulf turns to Hrothgar.}}

\textbf{Beowulf} I am eager to begin, great son of Healfdene.
Remember well, then, my wise lord,
provider of gold, what we agreed once before,
that if in your service it should so happen
that I am sundered from life, that you would assume the place
of a father towards me when I was gone.
Now extend your protection to the troop of my companions,
my young fellows, if the fight should take me;
convey also the gifts that you have granted to me,
beloved Hrothgar, to my lord Hygelac.
For on seeing this gold, the Geat chieftain,
Hrethel’s son, will perceive from its value
that I had met with magnificent patronage
from a giver of jewels, and that I had joy of him.

Let Unferth have the blade that I inherited
this wave-patterned sword
of rare hardness. With Hrunting shall I
achieve this deed – or death shall take me!’

\centerline{\textit{Snap blackout. Exit all apart from Beowulf}}

\newpage
\centerline{\textbf{Scene 4 - Grendel's Mother}}

\centerline{\textit{The lights now should give the impression of being underwater, with as little illumination as possible.}}

\textbf{The Bards} After these words the Weather-Geat prince
dived into the Mere – he did not care
to wait for an answer – and the waves closed over
the daring man. It was a day’s space almost
before he could glimpse ground at the bottom.
The grim and greedy guardian of the flood,
keeping her hungry hundred-season watch,
discovered at once that one from above,
a human, had sounded the home of the monsters.

She felt for the man and fastened upon him
her terrible hooks; but no harm came thereby
to the body within – the armour so ringed him
that she could not drive her dire fingers
through the mesh of the mail-shirt masking his limbs.
When she came to the bottom she bore him to her lair,
the mere-wolf, pinioning the mail-clad prince.

Then the man found
that he was in some enemy hall
where there was no water to weigh upon him
and the power of the flood could not pluck him away,
sheltered by its roof: a shining light he saw,
a bright fire blazing clearly.

\centerline{\textit{Lights up to reveal Beowulf and Grendel's Mother facing each other.}}

\centerline{\textit{Beowulf draws Hrunting and strikes, but it fails to do any damage. He throws it to the ground.}}
\centerline{\textit{He tackles Grendel's Mother, but as they fall she draws a dagger.}}
\centerline{\textit{She strikes at his back. His mail turns aside the blow.}}
\centerline{\textit{Beowulf then spots a giant's sword and takes it up.}}
\centerline{\textit{With one blow, he severs her head, killing her.}}

\centerline{\textit{At the moment of her death, the light changes to a clearer, brighter colour.}}
\centerline{\textit{Beowulf takes the sword and goes to find Grendel, then severs his head. Grendel's blood dissolves the blade.}}

\centerline{\textit{Beowulf disappears offstage.}}

\newpage
\centerline{\textbf{Scene 5 - The Hero Returns}}

\textbf{The Bards} Above, the wise men who watched with Hrothgar
the depths of the pool descried soon enough
blood rising in the broken water
and marbling the surface. 
The ninth hour had come; the keen-hearted Scyldings
abandoned the cliff-head; the kindly gold-giver
turned his face homeward. But the foreigners sat on,
staring at the pool with sickness at heart,
hoping they would look again on their beloved captain,
believing they would not.

The survivor of his enemies’ onslaught in battle
now set to swimming, and struck up through the water;
both the deep reaches and the rough wave-swirl
were thoroughly cleansed, now the creature from the otherworld
drew breath no longer in this brief world’s space.

Then the seamen’s Helm came swimming up
strongly to land, delighting in his sea-trove,
those mighty burdens that he bore along with him.
They went to meet him, a manly company,
thanking God, glad of their lord,
seeing him safe and sound once more.

The lake’s waters,
sullied with blood, slept beneath the sky.
They turned away from there and retraced their steps,
pacing the familiar paths back again
as bold as kings, carefree at heart.

The carrying of the head from the cliff by the Mere
was no easy task for any of them,
brave as they were. They bore it up,
four of them, on a spear, and transported back
Grendel’s head to the gold-giving hall.

Warrior-like they went, and it was not long
before they came, the seven bold Geats,
marching to the hall, and, among the company
walking across the land, their lord the tallest.

\newpage
\centerline{\textbf{Scene 6 - More Celebrations}}

\centerline{\textit{Beowulf \& Companions enter and walk up to the hall, carrying the head. They enter.}}
\centerline{\textit{Inside are all Hrothgar's people, talking. Beowulf walks up to address him.}}

\textbf{Beowulf} Behold! What you see here, O son of Healfdene,
prince of the Scyldings, was pleasant cargo for us:
– these trophies from the lake betoken victory!

Not easily did I survive
the fight under water; I performed this deed
not without a struggle. Our strife had ended
at its very beginning if God had not saved me.

Nothing could I perform in that fight with Hrunting,
it had no effect, fine weapon though it be.
But the Guide of mankind granted me the sight
of a huge Giant-sword hanging on the wall,
ancient and shining – and I snatched up the weapon.

When the hour afforded, in that fight I slew
the keepers of the hall. The coiling-patterned
blade burnt all away, as the blood sprang forth,
the hottest ever shed; the hilt I took from them.

So I avenged the violent slaughter
and outrages against the Danes; indeed it was fitting.

Now, I say, you may sleep in Heorot
free from care – your company of warriors
and every man of your entire people,
both the young men and the guard. Gone is the need
to fear those fell attacks of former times
on the lives of your earls, my lord of the Scyldings.

\centerline{\textit{He presents the hilt to Hrothgar, who studies it carefully. The hall falls silent}}

\textbf{Hrothgar} One who has tendered justice and truth to his people,
 their shepherd from of old, surely may say this,
remembering all that’s gone – that this man was born
to be the best of men. Beowulf, my friend,
your name shall resound in the nations of the earth
that are furthest away.

How wise you are to bear
your great strength so peaceably! I shall perform my vows
agreed in our forewords. It is granted to your people
that you shall live to be a long-standing comfort
and bulwark to the heroes.

Heremod was not so
for the honoured Scyldings, the sons of Edgewela:
his manhood brought not pleasure but a plague upon us,
death and destruction to the Danish tribes.
Inwardly his heart-hoard
grew raw and blood-thirsty; no rings did he give
to the Danes for his honour. And he dwelt an outcast,
paid the penalty for his persecution of them
by a life of sorrow. Learn from this, Beowulf:
study openhandedness! It is for your ears that I relate this,
and I am old in winters.

It is wonderful to recount
how in his magnanimity the Almighty God
deals out wisdom, dominion and lordship
among mankind. The Master of all things
will sometimes allow to the soul of a man
of well-known kindred to wander in delight:
He will grant him earth’s bliss in his own homeland,
the sway of the fortress-city of his people,
and will give him to rule regions of the world,
wide kingdoms: he cannot imagine,
in his unwisdom, that an end will come.
His life of bounty is not blighted by hint
of age or ailment; no evil care
darkens his mind, malice nowhere
bares the sword-edge, but sweetly the world
swings to his will; worse is not looked for.

At last his part of pride within him
waxes and climbs, the watchman of the soul
slumbering the while. That sleep is too deep,
tangled in its cares! Too close is the slayer
who shoots the wicked shaft from his bow!
For all his armour he is unable to protect himself:
the insidious bolt buries in his chest,
the crooked counsels of the accursed one.
What he has so long enjoyed he rejects as too little;
in niggardly anger renounces his lordly
gifts of gilt torques, forgets and misprises
his fore-ordained part, endowed thus by God,
the Master of Glory, with these great bounties.

And ultimately the end must come,
the frail house of flesh must crumble
and fall at its hour. Another then takes
the earl’s inheritance; open-handedly
he gives out its treasure, regardless of fear.

\newpage
Beloved Beowulf, best of warriors,
resist this deadly taint, take what is better,
your lasting profit. Put away arrogance,
noble fighter! The noon of your strength
shall last for a while now, but in a little time
sickness or a sword will strip it from you:
either enfolding flame or a flood’s billow
or a knife-stab or the stoop of a spear
or the ugliness of age; or your eyes’ brightness
lessens and grows dim. Death shall soon
have beaten you then, O brave warrior!

So it is with myself. I swayed the Ring-Danes
for fifty years here, defending them in war
with ash and with edge over the earth’s breadth
against many nations; until I numbered at last
not a single adversary beneath the skies’ expanse.
But what change of fortune befell me at my hearth
with the coming of Grendel; grief sprang from joy
when the old enemy entered our hall!
Great was the pain that persecution
thrust upon me. Thanks be to God,
the Lord everlasting, that I have lived until this day,
seen out this age of ancient strife
and set my gaze upon this gory head!

But join those who are seated, and rejoice in the feast,
O man clad in victory! We shall divide between us
many treasures when morning comes.

\centerline{\textit{Celebration. Beowulf sits and they begin to drink and feast again.}}

\newpage
\centerline{\textit{During The Bards's monologue, the lights fade through black, and come back up to daylight at the correct time.}}

\textbf{The Bards} Quite as before, the famous men,
guests of the hall, were handsomely feasted
on this new occasion. Then night’s darkness
grew on the company. The guard arose,
for their wise leader wished to rest,
the grey-haired Scylding. The Geat was ready enough
to go to his bed too, brave shieldsman.

The hero took his rest; the hall towered up
gilded, wide-gabled, its guest within sleeping
until the black raven blithe-hearted greeted
the heaven’s gladness. Hastening, the sunlight
shook out above the shadows. Sharp were the bold ones,
each atheling eager to set off,
back to his homeland: the high-mettled stranger
wished to be forging far in his ship.

That hardy man ordered Hrunting to be carried
back to the son of Edgelaf, bade him accept again
his well-loved sword; said that he accounted it
formidable in the fight, a good friend in war,
thanked him for the loan of it, without the least finding fault
with the edge of that blade; ample was his spirit!

By then the fighting-men were fairly armed-up
and ready for the journey; the Joy of the Danes went,
a prince, to the high seat where Hrothgar was,
one hero brave in battle hailed the other.

\newpage
\centerline{\textbf{Scene 7 - The Geats Take Their Leave}}
\centerline{\textit{Lights up. Beowulf and his companions are now stood facing Hrothgar and his people.}}

\textbf{Beowulf} We wish now to say, seafarers who
are come from afar, how keenly we desire
to return again to Hygelac. Here we were rightly,
royally, treated; you have entertained us well.

If I can ever on this earth earn of you,
O lord of men, more of your love
than I have so far done, by deeds of war,
I shall at once be ready. If ever I hear
that the neighbouring tribes intend your harm,
as those who hate you have done in the past,
I’ll bring a thousand thegns and heroes
here to help you. As for Hygelac, I know
that the Lord of the Geats, Guide of his flock,
young though he is, will yield his support
both in words and deeds so I may do you honour
and bring you a grove of grey-tipped spears
and my strength in aid when you are short of men.

\textbf{Hrothgar} These words you have delivered, the Lord in His wisdom
put in your heart. I have heard no man
of the age that you are utter such wisdom.

You are rich in strength and ripe of mind,
you are wise in your utterance. If ever it should happen
that spear or other spike of battle,
sword or sickness, should sweep away
the son of Hrethel, your sovereign lord,
shepherd of his people, my opinion is clear,
that the Sea-Geats will not be seeking for a better
man to be their king and keep their war-hoard,
if you still have life and would like to rule
the kingdom of your kinsmen. As I come to know
your temper, dear Beowulf, the better it pleases me.

You have brought it about that both the peoples,
the Sea-Geats and the Spear-Danes,
shall share out peace; the shock of war,
the old sourness, shall cease between us.

So long as I shall rule the reaches of this kingdom
we shall exchange wealth; a chief shall greet
his fellow with gifts over the gannet’s bath
as the ship with curved prow crosses the seas
with presents and pledges. Your people, I know,
always open-natured in the old manner,
are fast to friends and firm toward enemies.

\centerline{\textit{While Hrothgar speaks, Unferth brings twelve more gifts. Hrothgar presents them.}}

\textbf{Hrothgar} With these tokens of friendship,
take yourself safely back to your people
and return soon, my son.

\centerline{\textit{Hrothgar embraces Beowulf, and the Geats turn to leave.}}
\centerline{\textit{As The Bards speaks, the Geats exit and the light dims away from The Bards}}

\textbf{The Bards}  Beowulf went from him,
trod the green earth, a gold-resplendent warrior
rejoicing in his rings. Riding at anchor
the strayer of ocean stayed for her master.
Chiefly the talk returned as they walked
to Hrothgar’s giving. He was a king
blameless in all things, until old age at last,
that brings down so many, removed his proud strength.

They came then to the sea-flood, the spirited band
of warrior youth, wearing the ring-meshed
coat of mail.
The wide sea-boat with its soaring prow
was loaded at the beach there with battle-raiment,
with horses and arms. High rose the mast
above the lord Hrothgar’s hoard of gifts.

Out moved the boat then
to divide the deep water, left Denmark behind.
A special sea-dress, a sail, was hoisted
and belayed to the mast. The beams spoke.
The wind did not hinder the wave-skimming ship
as it ran through the seas, but the sea-going craft
with foam at its throat, furled back the waves,
her ring-bound prow planing the waters
till they caught sight of the cliffs of the Geats
and headlands they knew. The hull drove ahead,
urged by the breeze, and beached on the shore.

The harbour-guard was waiting at the water’s edge;
his eye had been scouring the stretches of the flood
in a long look-out for these loved men.
Now he moored the broad-ribbed boat in the sand,
held fast with hawsers, so no heft of the waves
should drive away again those darling timbers.

\newpage
He had the heroes’ hoard brought ashore,
their gold-plated armour. To go to their lord
was now but a step, to see again Hygelac
the son of Hrethel, at his home where he dwells
himself with his hero-band, hard by the sea-wall.

That was a handsome hall there. And high within it sat
a king of great courage. His consort was young,
but wise and discreet for one who had lived
so few years at court; the queen’s name was Hygd,
Hareth’s daughter. When she dealt out treasure
to the Geat nation, the gifts were generous,
there was nothing narrowly done.

The war-man himself came walking along
by the broad foreshores with his band of picked men,
trod the sea-beach. From the south blazed
the sun, the world’s candle. They carried themselves forward,
stepping on eagerly to the stronghold where
Ongentheow’s conqueror, the earls’ defender,
the warlike young king, was well-known for his
giving of neck-rings. The news of Beowulf’s
return was rapidly told to Hygelac
– that the shield of warriors, his own shoulder-companion,
had walked alive within the gates,
unscathed from the combat, and was coming to the hall.

\newpage
\centerline{\textbf{Scene 8 - Return to Hygelac's Court}}
\centerline{\textit{Lights up. Beowulf and his companions enter the hall of Hygelac and stand facing him.}}
\centerline{\textit{Also present are the other members of Hygelac's court.}}

\textbf{Beowulf} Hail Hygelac,
king of the Storm Geats, 
protector of your people, giver of rings.
Hail Hygd, lady of our people,
noble queen.
I am returned to you with honour and glory
from the land of the Scyldings.

\centerline{\textit{Hygd greets the Geats with the ceremonial wine-cup,}}
\centerline{\textit{offering it to each of them in turn. While she does this, she speaks: }}

\textbf{Hygd} What luck did you meet with, beloved Beowulf,
on your suddenly resolved seeking out
of distant strife over salt water,
battle at Heorot? Did you bring to that famous
leader Hrothgar some alleviation
of those woes so widely known? Overwhelming doubts
troubled my mind, mistrusting this voyage
of my dear liegeman. Long did I beg you
never to meet with this murderous creature
but to let the South Danes themselves bring an end
to their feud with Grendel. God be thanked
that safe and sound I see you here today!

\textbf{Beowulf} It has been told aloud, my lady Hygd,
and to many men by now, the meeting that there was
between myself and Grendel, the great time
we fought in that place where he had inflicted so much
grief and outrage, age-long disgrace
on the Victor-Scyldings. I avenged all.
No kinsman of Grendel shall have cause to take pride
in the sound that arose in the stretches of the night
– not the last of that alien and evil brood
on the face of the earth.

First I went in
to greet Hrothgar in the hall of the ring-giving.
As soon as the glorious son of Healfdene
knew my mind, he immediately
offered me a seat at his sons’ bench.
What hall-joys were there! A happier company
seated over mead I’ve not met with in my time
beneath the heavens. 

\newpage
A noble princess
fit to be the pledge of peace between nations
would move through the young men in the hall,
stirring their spirits; bestowing a torque
often upon a warrior before she went to her seat.

Later, when heaven’s jewel
had glided from the world, the wrathful creature,
dire dusk-fiend, came down to seek us out
where, still whole, we held the building.

The weight of the fight fell on Handscio,
the doomed blow came down on him; he died the first,
a warrior in his harness; the hero, my fellow,
was ground to death between Grendel’s jaws.

But the bloody-toothed slayer, bent on destruction,
was not going to go from that gold-giving hall
any the sooner: not empty-handed!
Proud of his might, he made proof of me,
groped out his greedy palm, but he could not best me
once I had stood up in anger against him.

Too long to repeat here how I paid back
the enemy of the people for his every crime;
but to your people, O my prince, my performance there
will bring honour. He broke away,
tasted life’s joys for a little while,
but his strong right hand stayed behind
in the hall of Heorot; humbled he went thence
and sank despairing in the depths of the Mere.

For this deadly fight the Friend of the Scyldings
recompensed me with plated gold,
a mort of treasure, when the morrow came
and we had benched ourselves at the banqueting table.

Thus we spent the space of a day there
seeking delight, until the ensuing dusk
came to mankind. Quick on its heels
the mother of Grendel moved to her revenge,
spurred on by sorrow; her son was death-taken
by Geat warspite. That gruesome she
avenged her son, struck down a warrior,
and boldly enough! The breath was taken
from the ancient counsellor, Ashhere, there.

\newpage
Nor could the Danish people, when day came,
give their death-wearied dear one to be burned,
escort him to the pyre: she had carried the body
to the mountain-torrent’s depth in her monstrous embrace.

This was for Hrothgar the harshest of the blows
that since so long had fallen on the leader of the people.
Distraught with care, the king then asked of me
a noble action – and in your name, Hygelac –
that I should risk my life among the rush of waters
and perform a great deed; he promised me reward.

Far and wide it is told how I found in the surges
the grim and terrible guardian of the deep.
After a hard hand-to-hand struggle
the whirlpool boiled with the blood of the mother;
I had hewn off her head in that hall underground
with a sword of huge size. I survived that fight
not without difficulty; but my doom was not yet.

The protector of warriors rewarded me
with a heap of treasure, Healfdene’s son.
The ways of that king accorded to usage:
I was not to forgo the gifts he had offered,
the meed of my strength; he bestowed upon me
the treasures I would have desired, the son of Healfdene;
now, brave king, I bring them to you.
I rejoice to present them. Joy, for me, always
lies in your gift. Little family
do I have in the world, Hygelac, besides yourself.

\centerline{\textit{He motions, and his companions present the standard, helmet, mail shirt and sword.}}

\textbf{Beowulf} Hrothgar gave me all this garb of war
with one word - that I should first relate to you whose legacy it is.
His brother Heorogar had it, he told me,
for a long while, as Lord of the Scyldings,
yet chose not to give this guardian of his breast
to his own son, the spirited Heoroweard,
friend though he was to him.
Flourish in the use of it!’

\centerline{\textit{Beowulf then presents Hygd with the necklace given him by Wealhtheow.}}

\textbf{Beowulf} My lady Hygd,
Lady Wealtheow of the Spear-Danes 
bestowed upon me this neck ring. 
I wish you joy of it!

\newpage
\centerline{\textit{Hygelac orders a sword to be brought in, and presents it to Beowulf.}}

\textbf{Hygelac} Excellent Beowulf!
The Geats have no more royal treasure to give
than this hilted sword, a treasured heirloom.
This I give to you, for your deeds of valour.

I also bestow on you an estate of seven thousand hides,
and with it a chief's stool and hall. 
More fitting reward for you service I cannot devise.

\textbf{The Bards} Such was the showing of the son of Edgetheow,
known for his combats and his courage in action.
His dealings were honourable: in drink he did not strike
at the slaves of his hearth; his heart was not savage.

The hero guarded well the great endowment
God had bestowed on him, a strength unequalled
among mankind. He had been misprised for long,
the sons of the Geats seeing little in him
and the lord of the Weather-Geats not willing to pay him
much in the way of honour on the mead-benches.

They firmly believed in his laziness - 
that the atheling was idle.
But for all such humblings
time brought reversal, and invested him with glory.

We will continue our tale,
but first let us pause,
and refresh ourselves from the mead-cup. 

\newpage
\subsection{Act 3}%%%%%%%%%%%%%%%%%%%%%%%%%%%%%%%%%%%%%%%%%%%%%%%%%%%%%%%%%%%%%

\centerline{\textbf{Scene 1 - The Dragon Awakens}}
\centerline{\textit{Lights up on The Bards}}

\textbf{The Bards}  Listen!
We have now told of Beowulf's journey to Denmark,
how he slew Grendel and Grendel's Mother,
how the brave Geat earned his glory.

He returned in triumph to Geatland,
came back to his lord Hygelac.

But it fell out after, in other days,
among the hurl of battle, when Hygelac lay dead
and the bills of battle had dealt death to his son Heardred
that the broad kingdom came by this turn
into Beowulf’s hands.

Half a century he ruled it well
the king had grown grey in the guardianship of the land 
until One began to put forth his power in the pitch-black night-times
the hoard-guarding \textit{Dragon} of a high barrow, raised above the moor.

Men did not know
of the way underground to it; but one man did enter,
went right inside, reached the treasure,
the heathen hoard, and his hand fell
on a golden goblet. The guardian, however,
if he had been caught sleeping by the cunning of the thief,
did not conceal this loss. It was not long till the near-
dwelling people discovered that the dragon was angry.

The causer of his pain had not purposed this;
it was without relish that he had robbed the hoard;
necessity drove him. The nameless slave
of one of the warriors, wanting shelter,
on the run from a flogging, had felt his way inside,
a sin-tormented soul. When he saw what was there
the intruder was seized with sudden terror;
but for all his fear, the unfortunate wretch
still took the golden treasure-cup.

\newpage
There were heaps of hoard-things in this hall underground
which once in gone days gleamed and rang;
the treasure of a race rusting derelict.

In another age an unknown man,
brows bent, had brought and hid here
the beloved hoard. The whole race
death-rapt, and of the ring of earls
one left alive; living on in that place
heavy with friend-loss, the hoard-guard
waited the same fate. His wit acknowledged
that the treasures gathered and guarded over the years
were his for the briefest while.

The Ravager of the night,
the burner who has sought out barrows from of old,
then found this hoard of undefended joy.
The smooth evil dragon swims through the gloom
enfolded in flame; the folk of that country
hold him in dread. He is doomed to seek out
hoards in the ground, and guard for an age there
the heathen gold: much good does it do him!

Thus for three hundred winters this waster of peoples
guarded underground the great hoard-hall
with his enormous might; until a man awoke
the anger in his breast by bearing to his master
the plated goblet as a peace-offering,
a token of new fealty. Thus the treasure was lightened
and the treasure-house was breached; the boon was granted
to the luckless slave, and his lord beheld
for the first time that work of a former race of men.

The waking of the worm awoke a new feud:
he glided along the rock, glared at the sight
of a foeman’s footprint: far too near his head
the intruder had stepped as he stole by him!

The treasure-guard eagerly
quartered the ground to discover the man
who had done him wrong during his sleep.
Seething with rage, he circled the barrow’s
whole outer wall, but no hint of a man
showed in the wilderness. Yet war’s prospect pleased him,
the thought of battle-action! He went back into the mound
to search for the goblet, and soon saw that one
of the tribe of men had tampered with the gold
of the glorious hoard.

\newpage
The hoard’s guardian
waited until evening only with difficulty.
The barrow-keeper was bursting with rage:
his fire would cruelly requite the loss
of the dear drinking-vessel.

At last day was gone,
to the worm’s delight; he delayed no further
inside his walls, but issued forth flaming,
armed with fire.

On every side the serpent’s ravages,
the spite of the foe, sprang to the eye –
how this hostile assailant hated and injured
the men of the Geats. Before morning’s light
he flew back to the hoard in its hidden chamber.
He had poured out fire and flame on the people,
he had put them to the torch; he trusted now to the barrow’s walls
and to his fighting strength. His faith misled him.

Beowulf was acquainted quickly enough
with the truth of the horror, for his own hall had itself
been swallowed in flame, the finest of buildings,
and the gift-stool of the Geats. Grief then struck
into his ample heart with anguished keenness.
The chieftain supposed he had sorely angered
the Ruler of all, the eternal Lord,
by breach of ancient law. His breast was thronged
with dark unaccustomed care-filled thoughts.

The fiery dragon’s flames had blasted
all the land by the sea, and its safe stronghold,
the fortress of the people. The formidable king
of the Geats now planned to punish him for this.

The champion of the fighting-men, chief of the earls,
gave orders for the making of a marvellous shield
worked all in iron; well he knew
that a linden shield would be of little service
– wood against fire. 

The distributor of rings disdained to go
with a troop of men or a mighty host
to seek the far-flier. He had no fear for himself
and discounted the worm’s courage and strength,
its prowess in battle. Battles in plenty
he had survived; valiant in all dangers,
he had come through many clashes since his cleansing of Heorot
and his extirpation of the tribe of Grendel.

The Lord of the Geats went with seven companions
to set eyes on the dragon; his anger rose in him.
He had by then discovered the cause of the attack
that had ravaged his people; the precious drinking-cup
had come into his hands from the hands of the informer.

He who had brought about the beginning of the feud
now made the ninth man in their company,
a miserable captive; cowed, he must show them
the way to the place, an unwilling guide.

For he alone knew the knoll and its earth-hall,
hard by the strand and the strife of the waves,
the underground hollow heaped to the roof
with intricate treasures. Attendant on the gold
was that underground ancient, eager as a wolf,
an awesome guardian; it was no easy bargain
for any mortal man to make himself its owner.

The stern war-king sat on the headland,
spoke encouragement to the companions of his hearth,
the gold-friend of the Geats. Gloomy was his spirit though,
death-eager, wandering; the fate was at hand
that was to overcome the old man there,
seek his soul’s hoard, and separate
the life from the body; not for long now
would the atheling’s life be lapped in flesh.

\newpage
\centerline{\textbf{Scene 2 - The Dragon}}

\centerline{\textit{The lights now reveal Beowulf, aged, standing on the headland with his companions.}}

\textbf{Beowulf} Many were the struggles I survived in youth
in times of danger; I do not forget them.
When that open-handed lord beloved by the people
received me from my father I was seven years old:
King Hrethel kept and fostered me,
gave me treasure and table-room, true to our kinship.
All his life he had as little hatred for me,
a warrior in hall, as he had for a son,
Herebeald, or Hathkin, or Hygelac my own lord.

I had the fortune in battle, by my bright sword,
to make return to Hygelac for the treasures he had given me.
He had granted me land, land to enjoy
and leave to my heirs. Little need was there
that Hygelac should go to the Gifthas or the Spear-Danes
or seek out ever in the Swedish kingdom
a weaker champion, and chaffer for his services.
I was always before him in the footing host,
by myself in the front.

Now shall hard edge,
hand and blade, do battle for the hoard!

Battles in plenty
I ventured in youth; and I shall venture this feud
and again achieve glory, the guardian of my people,
old though I am, if this evil destroyer
dares to come out of his earthen hall.’

\centerline{\textit{He addresses his companions}}

\textbf{Beowulf} I would choose not to take
any weapon to this worm, if I well knew
of some other fashion fitting to my boast
of grappling with this monster, as with Grendel before.
But as I must expect here the hot war-breath
of venom and fire, for this reason I have
my shield and armour. From the keeper of the barrow
I shall not flee one foot; but further than that
shall be worked out at the wall as our fate is given us
by the Creator of men. My mood is strong;
I forgo further words against the winged fighter.

\newpage
Men in armour! Your mail-shirts protect you:
await on the barrow the one of us two
who shall be better able to bear his wounds
after this onslaught. This affair is not for you,
nor is it measured to any man but myself alone
to match strength with this monstrous being,
attempt this deed. By daring will I
win this gold; war otherwise
shall take your king, terrible life’s-bane!

\centerline{\textit{Beowulf stands up and advances towards the barrow.}}
\centerline{\textit{As he approaches, smoke issues from the entrance. He roars a battle-cry.}}

\textbf{Beowulf} Come out of your hole, wyrm.
I have here a keen tempered blade for your hoard!

\centerline{\textit{He draws his sword.}}
\centerline{\textit{In response, a spurt of flame comes out, followed by the dragon. The companions shrink back.}}

\centerline{\textit{The dragon rushes towards Beowulf, who slashes it with his sword but fails to do much damage.}}
\centerline{\textit{The dragon backs off a little and breathes fire towards Beowulf, who defends himself with his shield.}}
\centerline{\textit{Throughout the following speech, Beowulf and the Dragon continue to fight. Beowulf should appear to be losing.}}
\centerline{\textit{As the pair separate for a while to catch their breath, focus switches to Wiglaf and the other companions, who are hiding.}}

\textbf{Wiglaf} I remember the time, as we were taking mead
in the banqueting hall, when we bound ourselves
to the gracious lord who granted us arms,
that we would make return for these trappings of war,
these helms and hard swords, if an hour such as this
should ever chance for him. 

That day has now come
when he stands in need of the strength of good fighters,
our lord and liege. Let us go to him,
help our leader for as long as it requires,
the fearsome fire-blast. I had far rather
that the flame should enfold my flesh-frame there
alongside my gold-giver – as God knows of me.

To bear our shields back to our homes
would seem unfitting to me, unless first we have been able
to kill the foe and defend the life
of the prince of the Weather-Geats. I well know
that former deeds deserve not that, alone
of the flower of the Geats, he should feel the pain,
sink in the struggle; sword and helmet,
shield and mail-shirt, shall be our common gear.

\newpage
\centerline{\textit{Wiglaf advances on the dragon, drawing his sword. The other companions stay where they are.}}

\textbf{Wiglaf} 
Beloved Beowulf, bear all things well!
Help is at hand!

\centerline{\textit{The dragon attacks again, and Wiglaf's shield is burnt up by the dragon. He shelters behind Beowulf's shield.}}
\centerline{\textit{Beowulf strikes again, but his sword breaks. The dragon once again backs off a short way.}}

\centerline{\textit{The dragon rushes in again, and bites Beowulf in the neck. Wiglaf strikes the dragon in the neck.}}
\centerline{\textit{Beowulf then takes his langseax and stabs it through the dragon's body, killing it.}}

\newpage
\centerline{\textbf{Scene 3 - Beowulf's Death}}
\centerline{\textit{Beowulf walks over to a ledge and sits down, followed by Wiglaf. He is clearly in pain.}}
\centerline{\textit{Wiglaf removes Beowulf's helmet and attends to his wounds, washing them.}}

\textbf{Beowulf} I would now wish to give my garments in battle
to my own son, if any such
after-inheritor, an heir of my body,
had been granted to me. I have guarded this people
for half a century; not a single ruler
of all the nations neighbouring about
has dared to affront me with his friends in war,
or threaten terrors. What the times had in store for me
I awaited in my homeland; I held my own,
sought no secret feud, swore very rarely
a wrongful oath.

In all of these things,
sick with my life’s wound, I may still rejoice:
for when my life shall leave my body
the Ruler of Men may not charge me
with the slaughter of kinsmen.

Quickly go now,
beloved Wiglaf, and look upon the hoard
under the grey stone, now the serpent lies dead,
sleeps rawly wounded, bereft of his treasure.
Make haste, that I may gaze upon that golden inheritance,
that ancient wealth; that my eyes may behold
the clear skilful jewels: more calmly then may I
on the treasure’s account take my departure
of life and of the lordship I have long held.

\centerline{\textit{Wiglaf disappears into the cave, and returns with an armful of treasure.}}
\centerline{\textit{Meanwhile, Beowulf is clearly dying. Wiglaf, on returning, 
sprinkles water on Beowulf in an attempt to revive him.}}

\newpage
\textbf{Beowulf} I wish to put in words my thanks
to the King of Glory, the Giver of All,
the Lord of Eternity, for these treasures that I gaze upon,
that I should have been able to acquire for my people
before my death-day an endowment such as this.

My life’s full portion I have paid out now
for this hoard of treasure; you must attend the people’s
needs henceforward; no further may I stay.

Bid men of battle build me a tomb
fair after fire, on the foreland by the sea
that shall stand as a reminder of me to my people,
towering high above Hronesness
so that ocean travellers shall afterwards name it
Beowulf’s barrow, bending in the distance
their masted ships through the mists upon the sea.

\centerline{\textit{Beowulf slowly unclasps the collar from his neck, and passes it to Wiglaf.}}

\textbf{Beowulf} You are the last man left of our kindred,
the house of the Waymundings! fate has lured
each of my family to his fated end,
each earl through his valour; I must follow them.

\centerline{\textit{Beowulf dies.}}

\newpage
\centerline{\textbf{Scene 4 - Grief}}
\centerline{\textit{Wiglaf sits beside Beowulf's body, trying to wake him until it becomes obvious that it is futile.}}
\centerline{\textit{Shamefully, the other companions approach.}}

\textbf{Wiglaf} A man who would speak the truth may say with justice
that a lord of men who allowed you those treasures,
who bestowed on you the trappings that you stand there in
had quite thrown away and wasted cruelly
all that battle-harness when the battle came upon him.

The king of our people had no cause to boast
of his companions of the guard. Yet God vouchsafed him,
the Master of Victories, that he should avenge himself
when courage was wanted, by his weapon alone.
I was little equipped to act as body-guard
for him in the battle, but, above my own strength,
I began all the same to support my kinsman.

Our deadly enemy grew ever the weaker –
when I had struck him with my sword – less strongly welled
the fire from his head. Too few supporters
flocked to our prince when affliction came.

Now there shall cease for your kin the receiving of treasure,
the bestowal of swords, all satisfaction of ownership,
all comfort of home. Your kinsmen every one,
shall become wanderers without land-rights
as soon as athelings over the world
shall hear the report of how you fled,
a deed of ill fame. Death is better
for any earl than an existence of disgrace!

\centerline{\textit{During the following passage, the disgraced companions exit.}}

\textbf{The Bards} He bade that the combat’s result be proclaimed in the city
over the brow of the headland; there the band of earls
had sat all morning beside their shields
in heavy spirits, half expecting
that it would be the last day of their beloved man,
half hoping for his return. The rider from the headland
in no way held back the news he had to tell;
as his commission was, he called out over all:

\newpage
\textbf{The Messenger} The Lord of the Geats lies now on his slaughter-bed,
the leader of the Weathers, our loving provider,
dwells in his death-rest through the dragon’s power.
Stretched out beside him, stricken with the knife,
lies his deadly adversary. With the edge of the sword
he could not contrive, try as he might,
to wound the monster. Weoxstan’s son
Wiglaf abides with Beowulf there,
one earl waits on the other one lifeless;
in weariness of heart he watches by the heads
of friend and foe.

Haste is best now,
that we should go to look on the lord of the people,
then bring our ring-bestower on his road,
escort him to the pyre. More than one portion of wealth
shall melt with the hero, for there’s a hoard of treasure
and gold uncounted; a grim purchase,
for in the end it was with his own life
that he bought these rings.

\centerline{\textit{A solemn procession makes it's way to towards Beowulf's body.}}

\textbf{The Bards} Such was the rehearsal of the hateful tidings
by that bold messenger; amiss in neither
words nor facts. The war-band arose;
they went unhappily under Earna-ness
to look on the wonder with welling tears.
They found him on the sand, his soul fled,
keeping his resting-place: rings he had given them
in former times! But the final day
had come for the champion; and the chief of the Geats,
the warrior-king, had met his wondrous death.

\textbf{Wiglaf} Many must often endure distress
for the sake of one; so it is with us.
We could not urge any reason
on our beloved king, the keeper of the land,
why he should not approach the protector of the gold
but let him lie where he had long been already
and abide in his den until the end of the world.
He held to his high destiny.

The hoard has been seen
that was acquired at such a cost; too cruel the fate
that impelled the king of the people towards it!
I myself was inside there, and saw all
the wealth of the chamber once my way was open
– little courtesy was shown in allowing me to pass
beneath the earth-wall.

I urgently filled
my hands with a huge heap of the treasures
stored in the cave, carried them out
to my lord here. He was alive still
and commanded his wits. Much did he say
in his grief, the old man; he asked me to speak to you,
ordered that on the place of the pyre you should raise
a barrow fitting your friend’s achievements;
conspicuous, magnificent, as among men he was
while he could wield the wealth of his stronghold
the most honoured of warriors on the wide earth.

Let us now hasten to behold again,
and approach once more that mass of treasures,
awesome under the walls; I shall guide you,
so that from near at hand you may behold sufficiently
the thick gold and the bracelets. Let a bier be
made ready,
contrive it quickly, so that when we come out again
we may take up our king, carry the man
beloved by us to his long abode
where he must rest in the Ruler’s keeping.

\textbf{The Bards} Then the son of Weoxstan, worthy in battle,
had orders given to owners of homesteads
and a great many warriors, that the governors of the people
from far and wide should fetch in wood
for the hero’s funeral pyre.

\textbf{Wiglaf} Now the flames shall grow dark
and the fire destroy the sustainer of the warriors
who often endured the iron shower
when, string-driven, the storm of arrows
sang over shield-wall, and the shaft did its work,
sped by its feathers, furthered the arrow-head.

\textbf{The Bards} Then in his wisdom Weoxstan’s son
called out from the company of the king’s own thegns
seven men in all, who excelled among them,
and, himself the eighth warrior, entered in beneath
that unfriendly roof. The front-stepping man
bore in his hand a blazing torch.

\newpage
\centerline{\textit{Wiglaf and some of the people enter the barrow. Meanwhile, Beowulf's body is laid on a bier.}}

\textbf{The Bards} When the men perceived a piece of the hoard
that remained unguarded, mouldering there
on the floor of the chamber, they did not choose by lot
who should remove it; undemurring,
as quickly as they could, they carried outside
the precious treasure.

\centerline{\textit{The people construct a funeral pyre and place Beowulf on top of it.}}

\textbf{The Bards} The Geat race then reared up for him
a funeral pyre. It was not a petty mound,
but shining mail-coats and shields of war
and helmets hung upon it, as he had desired.
Then the heroes, lamenting, laid out in the middle
their great chief, their cherished lord.

On top of the mound the men then kindled
the biggest of funeral-fires. Black wood-smoke
arose from the blaze, and the roaring of flames
mingled with weeping. The winds lay still
as the heat at the fire’s heart consumed
the house of bone. And in heavy mood
they uttered their sorrow at the slaughter of their lord.

\centerline{\textit{The fire is lit. Smoke begins to rise.}}

A woman of the Geats in grief sang out
the lament for his death. Loudly she sang,
her hair bound up, the burden of her fear
that evil days were destined her
– troops cut down, terror of armies,
bondage, humiliation. Heaven swallowed the smoke.

Then the Storm-Geat nation constructed for him
a stronghold on the headland, so high and broad
that seafarers might see it from afar.
The beacon to that battle-reckless man
they made in ten days. What remained from the fire
they cast a wall around, of workmanship
as fine as their wisest men could frame for it.

\centerline{\textit{The fire grows.}}

\newpage
They placed in the tomb both the torques and the jewels,
all the magnificence that the men had earlier
taken from the hoard in hostile mood.
They left the earls’ wealth in the earth’s keeping,
the gold in the dirt. It dwells there yet,
of no more use to men than in ages before.

Then the warriors rode around the barrow,
twelve of them in all, athelings’ sons.
They recited a dirge to declare their grief,
spoke of the man, mourned their king.

They praised his manhood and the prowess of his hands,
they raised his name; it is right a man
should be lavish in honouring his lord and friend,
should love him in his heart when the leading-forth
from the house of flesh befalls him at last.

This was the manner of the mourning of the men of the Geats,
sharers in the feast, at the fall of their lord:
they said that he was of all the world’s kings
the gentlest of men, and the most gracious,
the kindest to his people, the keenest for fame.

\centerline{\textit{End.}}

\end{document}
