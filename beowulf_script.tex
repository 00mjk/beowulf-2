\documentclass[a4paper]{article}

% Table of contents depth (currently unused)
\setcounter{tocdepth}{3}

% Section numbering depth (zero for no numbering)
\setcounter{secnumdepth}{0}

% latex package inclusions here
%\usepackage{fullpage}
\usepackage{hyperref}
\usepackage{tabulary}
%\usepackage{amsthm}

% set up BNF generator
%\usepackage{syntax}
%\setlength{\grammarparsep}{10pt plus 1pt minus 1pt}
%\setlength{\grammarindent}{10em} 

% set up source code inclusion
\usepackage{listings}
\lstset{
  tabsize=2,
  basicstyle = \ttfamily\small,
  columns=fullflexible
}
% Usage for the above like so:
% \begin{lstlisting}
%   CODE CODE CODE
% \end{lstlisting}

% in-line code styling (same style as listing)
\newcommand{\shell}[1]{\lstinline{#1}}

%%%%%%%%%%%%%%%%%%%%%%%%%%%%%%%%%%%%%%%%%%%%%%%%%%%%%%%%%%%%%%%%%%%%%%%%%%%%%%%

\begin{document}
\title{Beowulf Script}
\date{2016}
\author{
Daniel Clay \\ 
}
\maketitle

%%%%%%%%%%%%%%%%%%%%%%%%%%%%%%%%%%%%%%%%%%%%%%%%%%%%%%%%%%%%%%%%%%%%%%%%%%%%%%%
\section{Introduction}
%%%%%%%%%%%%%%%%%%%%%%%%%%%%%%%%%%%%%%%%%%%%%%%%%%%%%%%%%%%%%%%%%%%%%%%%%%%%%%%

\subsection{Characters}%%%%%%%%%%%%%%%%%%%%%%%%%%%%%%%%%%%%%%%%%%%%%%%%%%%%%%%%

In order of appearance:
\begin{description}
    \item[The Bard] Narrator of the play 
    \item[Hrothgar] King of the Spear-Danes (Scyldings)
    \item[Grendel] An evil monster, descendant of Cain
    \item[Beowulf] A hero of the Geatish people
    \item[Beowulf's companions] 14 Geatish thegns
    \item[Lookout] A man of the Spear Danes who keeps watch for ships
    \item[Wulfgar] Hrothgar's Counsellor
    \item[Unferth] Hrothgar's spokesman
    \item[Wealhtheow] Queen of the Spear-Danes
    \item[Hrethic \& Hrothmund] Sons of Hrothgar
    \item[Grendel's Mother] A monstrous ogress
    \item[Ashhere] Hrothgar's Counsellor
    \item[Hygelac] King of the Geats
    \item[Hygd] Queen of the Geats
    \item[The Dragon] A dragon
    \item[A Slave] Who awakens the dragon
    \item[Wiglaf] Beowulf's companion
\end{description}

Of the cast, most of Beowulf's companions can be omitted as required, and those
present at the court of Hrothgar can be doubled up with those present at the court
of Hygelac. 

Most of the minor roles can be played by men or women, as can Grendel, Grendel's
Mother, the Dragon and the Bard

\subsection{Staging}%%%%%%%%%%%%%%%%%%%%%%%%%%%%%%%%%%%%%%%%%%%%%%%%%%%%%%%%%%%

This play works best when some of the character of the mead-hall, for which the
original work was composed, is retained. Staging should be minimal and as far as
possible abstracted away. No set changes should be required.

Lighting by open flame is preferred to lighting by electric light.

\subsection{Structure of the play}%%%%%%%%%%%%%%%%%%%%%%%%%%%%%%%%%%%%%%%%%%%%%

Despite the length of the original poem, which runs to over 3000 lines, the play
should be able to be performed straight through, without and interval.

The poem itself is characterised by the three combats Beowulf partakes in; that
with Grendel, with Grendel's Mother, and with the Dragon. 

%%%%%%%%%%%%%%%%%%%%%%%%%%%%%%%%%%%%%%%%%%%%%%%%%%%%%%%%%%%%%%%%%%%%%%%%%%%%%%%
\section{Script}
%%%%%%%%%%%%%%%%%%%%%%%%%%%%%%%%%%%%%%%%%%%%%%%%%%%%%%%%%%%%%%%%%%%%%%%%%%%%%%%

Listen! Beowulf kills Grendel, then Grendel's Mother, then a dragon. 

\end{document}
